% Options for packages loaded elsewhere
\PassOptionsToPackage{unicode}{hyperref}
\PassOptionsToPackage{hyphens}{url}
%
\documentclass[
  10pt,
]{article}
\usepackage{lmodern}
\usepackage{amssymb,amsmath}
\usepackage{ifxetex,ifluatex}
\ifnum 0\ifxetex 1\fi\ifluatex 1\fi=0 % if pdftex
  \usepackage[T1]{fontenc}
  \usepackage[utf8]{inputenc}
  \usepackage{textcomp} % provide euro and other symbols
\else % if luatex or xetex
  \usepackage{unicode-math}
  \defaultfontfeatures{Scale=MatchLowercase}
  \defaultfontfeatures[\rmfamily]{Ligatures=TeX,Scale=1}
\fi
% Use upquote if available, for straight quotes in verbatim environments
\IfFileExists{upquote.sty}{\usepackage{upquote}}{}
\IfFileExists{microtype.sty}{% use microtype if available
  \usepackage[]{microtype}
  \UseMicrotypeSet[protrusion]{basicmath} % disable protrusion for tt fonts
}{}
\makeatletter
\@ifundefined{KOMAClassName}{% if non-KOMA class
  \IfFileExists{parskip.sty}{%
    \usepackage{parskip}
  }{% else
    \setlength{\parindent}{0pt}
    \setlength{\parskip}{6pt plus 2pt minus 1pt}}
}{% if KOMA class
  \KOMAoptions{parskip=half}}
\makeatother
\usepackage{xcolor}
\IfFileExists{xurl.sty}{\usepackage{xurl}}{} % add URL line breaks if available
\IfFileExists{bookmark.sty}{\usepackage{bookmark}}{\usepackage{hyperref}}
\hypersetup{
  pdftitle={Le vocabulaire},
  pdfauthor={Adrien Méli},
  hidelinks,
  pdfcreator={LaTeX via pandoc}}
\urlstyle{same} % disable monospaced font for URLs
\usepackage[margin=1in]{geometry}
\usepackage{longtable,booktabs}
% Correct order of tables after \paragraph or \subparagraph
\usepackage{etoolbox}
\makeatletter
\patchcmd\longtable{\par}{\if@noskipsec\mbox{}\fi\par}{}{}
\makeatother
% Allow footnotes in longtable head/foot
\IfFileExists{footnotehyper.sty}{\usepackage{footnotehyper}}{\usepackage{footnote}}
\makesavenoteenv{longtable}
\usepackage{graphicx}
\makeatletter
\def\maxwidth{\ifdim\Gin@nat@width>\linewidth\linewidth\else\Gin@nat@width\fi}
\def\maxheight{\ifdim\Gin@nat@height>\textheight\textheight\else\Gin@nat@height\fi}
\makeatother
% Scale images if necessary, so that they will not overflow the page
% margins by default, and it is still possible to overwrite the defaults
% using explicit options in \includegraphics[width, height, ...]{}
\setkeys{Gin}{width=\maxwidth,height=\maxheight,keepaspectratio}
% Set default figure placement to htbp
\makeatletter
\def\fps@figure{htbp}
\makeatother
\setlength{\emergencystretch}{3em} % prevent overfull lines
\providecommand{\tightlist}{%
  \setlength{\itemsep}{0pt}\setlength{\parskip}{0pt}}
\setcounter{secnumdepth}{5}
\usepackage{fontawesome}
\usepackage{amsmath}
\usepackage{tipa}
\usepackage{moodle}
\usepackage{hyperref}
\usepackage{booktabs}
\usepackage[utf8]{inputenc}
\usepackage[light,sfdefault]{roboto}
\usepackage{titlesec}
\usepackage{multicol}
%\usepackage{tufte-book}
%\usepackage{fourier}
%\usepackage{montserrat}
%\usepackage[T1]{fontenc}
%\usepackage[french]{babel}

%----------------------------------------------------------------------------------------
%	DEFINE COLOURS
%----------------------------------------------------------------------------------------
%\definecolor{darkPrimaryColor}{HTML}{303F9F}
\definecolor{darkPrimaryColor}{HTML}{2c5272}
\definecolor{primaryColor}{HTML}{3F51B5}
\definecolor{lightPrimaryColor}{HTML}{C5CAE9}
\definecolor{textPrimaryColor}{HTML}{FFFFFF}
\definecolor{accentColor}{HTML}{FF5252}
\definecolor{primaryTextColor}{HTML}{212121}
\definecolor{secondaryTextColor}{HTML}{757575}
\definecolor{dividerColor}{HTML}{BDBDBD}

% -----------------------------------------------------------------
% Hyper Setup
% -----------------------------------------------------------------
\hypersetup{
    %bookmarks=true,         % show bookmarks bar?
    unicode=false,          % non-Latin characters in Acrobat�s bookmarks
    pdftoolbar=true,        % show Acrobat�s toolbar?
    pdfmenubar=true,        % show Acrobat�s menu?
    pdffitwindow=false,     % window fit to page when opened
    pdfstartview={FitH},    % fits the width of the page to the window
    %pdftitle={},    % title
    pdfauthor={Adrien Meli},     % author 
    pdfsubject={Phonological rules},   % subject of the document
    pdfcreator={Creator},   % creator of the document
    pdfproducer={Producer}, % producer of the document
    pdfkeywords={Second Language Acquisition, French, English, phonology}, % list of keywords
    pdfnewwindow=true,     % links in new window
    colorlinks=true,       % false: boxed links; true: colored links
    linkcolor=black,          % color of internal links
    citecolor=black,        % color of links to bibliography
    filecolor=magenta,      % color of file links
    urlcolor=blue,           % color of external links
    bookmarksopen=false,
    anchorcolor=black,
    bookmarksnumbered=true,
    pdfpagemode=UseOutlines,    %None/UseOutlines/UseThumbs/FullScreen
    linktocpage=true
}

% ********************Captions and Hyperreferencing / URL **********************

% Captions: This makes captions of figures use a boldfaced small font.

\usepackage[margin=10pt,font=small,labelfont=bf,labelsep=endash]{caption}

% -----------------------------------------------------------------
% TABLE OF CONTENTS
% -----------------------------------------------------------------
\usepackage[dotinlabels]{titletoc}
\titlecontents{section}[0em] % entries are pushed to the rtight
  {} % code to change the appearance
  {} % section number: increase distance to push to the left
  {\hspace*{3.3em}}
  {\titlerule*[1.9mm]{.}\contentspage}
% remove subsections from TOC
\setcounter{tocdepth}{1}

% multi-line curly brackets

\newenvironment{rightbracedtext}
 {$\kern-\nulldelimiterspace\left.\begin{tabular}{@{}l@{}}}
 {\end{tabular}\right\}$}

\newenvironment{leftbracedtext}{$\left\{\begin{tabular}{@{}l}}{\end{tabular}\right.$}

\titleformat{\chapter}[display]
  {\normalfont\bfseries\filcenter}{\LARGE\thechapter}{1ex}
  {\titlerule[2pt]\vspace{2ex}}[\vspace{1ex}{\titlerule[2pt]}]
\titleformat{name=\chapter,numberless}[display]
  {\normalfont\LARGE\bfseries\filcenter}{}{1ex}
  {\vspace{2ex}}[\vspace{1ex}]

%----------------------------------------------------------------------------------------
%	MODIFY SECTION STYLES
%----------------------------------------------------------------------------------------

%\usepackage{titlesec} % Required for modifying sections
%
%%------------------------------------------------
%% Section
%
%\titleformat
%   {\section} % Section type being modified
%   [block] % Shape type, can be: hang, block, display, runin, leftmargin, rightmargin, drop, wrap, frame
%   {\centering\bfseries\color{darkPrimaryColor}} % Format of the whole section
%   {} % Format of the section label
%   {0pt} % Space between the title and label
%   %{\titlerule\newline{\thetitle. }} % Code before the label
%   {\titlerule\newline} % Code before the label
%   [\titlerule]% after code
%
%\titlespacing{\section}
%   {0pt} % left
%   {\baselineskip} % before title
%   {\baselineskip} % Spacing around section titles, the order is: left, before and after
%
%%------------------------------------------------
%% Subsection
%
%\titleformat
%   {\subsection}
%   [block] % Shape type, can be: hang, block, display, runin, leftmargin, rightmargin, drop, wrap, frame
%   {\itshape \color{accentColor}\bfseries}% format
%   %{\thesubsection.}% label
%   {}% label
%   {6pt}% sep btw label and title
%   {}% before code
%   []% after code
% 
%\titlespacing{\subsection}
%   {6pt}% left
%   {\baselineskip} % before title
%   {\baselineskip} % Spacing around section titles, the order is: left, before and after
%
%%------------------------------------------------
%% Subsubsection
%
%\titleformat
%   {\subsubsection}
%   [block] % Shape type, can be: hang, block, display, runin, leftmargin, rightmargin, drop, wrap, frame
%   {\ttfamily \bfseries \itshape }% format
%   {\thesubsubsection}% label
%   {4pt}% sep btw label and title
%   {}% before code
%   []% after code
% 
%\titlespacing{\subsubsection}
%   {12pt}% left
%   {\baselineskip} % before title
%   {\baselineskip}% after sep
%
%%\renewcommand\thesubsection{(\alph{subsection})}
%
%% from bookdown to create column environments
%\newenvironment{columns}[1][]{}{}
%
%\newenvironment{column}[1]{\begin{minipage}{#1}\ignorespaces}{%
%\end{minipage}
%\ifhmode\unskip\fi
%\aftergroup\useignorespacesandallpars}
%
%\def\useignorespacesandallpars#1\ignorespaces\fi{%
%#1\fi\ignorespacesandallpars}
%
%\makeatletter
%\def\ignorespacesandallpars{%
%  \@ifnextchar\par
%    {\expandafter\ignorespacesandallpars\@gobble}%
%    {}%
%}
%\makeatother
\ifluatex
  \usepackage{selnolig}  % disable illegal ligatures
\fi

\title{Le vocabulaire}
\author{Adrien Méli}
\date{October 07, 2020}

\begin{document}
\maketitle

{
\setcounter{tocdepth}{1}
\tableofcontents
}
Comment constituer une liste de vocabulaire anglais ?

Le vocabulaire s'apprend plus efficacement si la liste de mots reflète certaines caractéristiques grammaticales.

Voici pour les principales catégories grammaticales quelques conseils sur les informations à inclure dans votre liste de vocabulaire.
Ces conseils sont assortis de quelques rappels et astuces.

\hypertarget{les-noms-anglais}{%
\section{Les noms anglais}\label{les-noms-anglais}}

En anglais, certains noms peuvent se mettre au pluriel, d'autres ne le peuvent pas.

Les premiers sont dits ``\textbf{comptables}'', les deuxièmes, ``\textbf{incomptables}''.

\hypertarget{les-noms-comptables}{%
\subsection{Les noms comptables}\label{les-noms-comptables}}

Ces noms prennent obligatoirement un article au singulier.
Ils se mettent généralement au pluriel avec le suffixe \textless-(e)s\textgreater.

\begin{quote}
\textbf{Conseil :} Dans votre liste de vocabulaire, faites précéder les noms comptables de l'article indéfini ``\emph{a}''.

\emph{Exemple : ``a tool'' \(\rightarrow\)} ``un outil''.
\end{quote}

\hypertarget{les-noms-incomptables}{%
\subsection{Les noms incomptables}\label{les-noms-incomptables}}

Ces noms ne peuvent pas se mettre au pluriel.

\begin{quote}
\textbf{Conseil :} Dans votre liste de vocabulaire, indiquez les noms incomptables par ``\emph{(U)}'', pour ``\emph{uncountable}''.

\emph{Exemple : ``evidence (U)'' \(\rightarrow\)} ``des preuves''.
\end{quote}

\hypertarget{bon-uxe0-savoir}{%
\subsection{Bon à savoir}\label{bon-uxe0-savoir}}

\hypertarget{mots-courants-incomptables}{%
\subsubsection{Mots courants incomptables}\label{mots-courants-incomptables}}

Attention aux noms suivants, qui ne se mettent pas au pluriel en anglais :

\begin{itemize}
\tightlist
\item
  \emph{information (U)}
\item
  \emph{furniture (U)} (``meuble'')
\item
  \emph{advice (U)} (``conseil'')
\end{itemize}

Pour dire ``un meuble'' ou ``un conseil'', on dira ``\emph{a piece of furniture}'' ou ``\emph{a piece of advice}''.

\hypertarget{singuliers-avec--s}{%
\subsubsection{Singuliers avec \textless-s\textgreater{}}\label{singuliers-avec--s}}

Les mots suivants ont un \textless-s\textgreater{} au singulier :

\begin{itemize}
\tightlist
\item
  \emph{a series} (``une série'')
\item
  \emph{a means} (``un moyen'')
\item
  \emph{a species} (``une espèce'')
\end{itemize}

Attention aussi à \color[HTML]{f44336}\emph{the news} \color{black} (``les nouvelles''), qui est \textbf{toujours singulier}.

\hypertarget{pluriels-sans--s}{%
\subsubsection{Pluriels sans \textless-s\textgreater{}}\label{pluriels-sans--s}}

Inversement, il existe des noms qui peuvent ou doivent se conjuguer au pluriel :

\begin{itemize}
\tightlist
\item
  \color[HTML]{f44336}\emph{the police} \color{black} est \textbf{toujours pluriel}.
\item
  \emph{staff}, \emph{team} ou \emph{family} peuvent se conjuguer au pluriel.
\end{itemize}

\hypertarget{les-adjectifs}{%
\section{Les adjectifs}\label{les-adjectifs}}

Les adjectifs anglais sont \textbf{invariables}, et se placent \textbf{avant} le nom.

\begin{itemize}
\tightlist
\item
  ``Des maisons bleues'' \(\rightarrow\) \emph{blue houses}.
\end{itemize}

\begin{quote}
\textbf{Conseil :} Dans votre liste de vocabulaire, indiquez bien la préposition avec laquelle l'adjectif se construit.

\emph{Exemple : ``similar \textbf{to}'' \(\rightarrow\)} ``semblable à''.
\end{quote}

Voici quelques exemples d'adjectifs fréquents se construisant avec des prépositions différentes du français :

\begin{itemize}
\tightlist
\item
  \emph{responsible \textbf{for}} \(\rightarrow\) ``responsable de''
\item
  \emph{dependent \textbf{on}} \(\rightarrow\) ``dépendant de''
\item
  \emph{addicted \textbf{to}} \(\rightarrow\) ``accroc à''
\item
  \emph{interested \textbf{in}} \(\rightarrow\) ``intéressé par''
\item
  \emph{good \textbf{at}} \(\rightarrow\) ``bon en''
\item
  \emph{different \textbf{from}} \(\rightarrow\) ``différent de''
\item
  \emph{worried \textbf{about}} \(\rightarrow\) ``inquiet de''
\item
  \emph{surprised \textbf{at}} \(\rightarrow\) ``surpris par''
\end{itemize}

\hypertarget{les-verbes}{%
\section{Les verbes}\label{les-verbes}}

\begin{quote}
\textbf{Conseil :} Dans votre liste de vocabulaire, faites précéder les verbes de ``\emph{to}'' afin de les distinguer des autres catégories grammaticales.

\emph{Exemple : ``to row'' \(\rightarrow\)} ``ramer''
\end{quote}

Comme les adjectifs, les verbes s'apprennent avec leur \textbf{construction}.

\hypertarget{les-verbes-transitifs}{%
\subsection{Les verbes transitifs}\label{les-verbes-transitifs}}

On distingue les \textbf{verbes transitifs directs}, qui admettent un complément d'objet direct, des \textbf{verbes transitifs indirects}, dont le complément
est séparé du verbe par une préposition.

Considérez les exemples suivants :

\begin{itemize}
\tightlist
\item
  \emph{to listen \textbf{to} stg} \(\rightarrow\) ``écouter qqch''
\item
  \emph{to look \textbf{for} stg} \(\rightarrow\) ``chercher qqch''
\item
  \emph{to look \textbf{at} stg} \(\rightarrow\) ``regarder qqch''
\end{itemize}

Dans ces exemples, le verbe anglais est transitif indirect, tandis que le verbe français est transitif direct.

À l'inverse :

\begin{itemize}
\tightlist
\item
  \emph{to obey sb} \(\rightarrow\) ``obéir \textbf{à} qn''
\end{itemize}

\begin{quote}
\textbf{Conseil :} Dans votre liste de vocabulaire, indiquez la construction du verbe avec ``\emph{stg}'' (``\emph{something}'') ou ``\emph{sb}'' (``\emph{somebody}'').

\emph{Exemple : ``to blame sb for stg'' \(\rightarrow\)} ``reprocher qqch à qn''
\end{quote}

\hypertarget{les-verbes-intransitifs}{%
\subsection{Les verbes intransitifs}\label{les-verbes-intransitifs}}

Ces verbes se contruisent sans complément, et par conséquent ne peuvent se mettre au passif.

\emph{Exemples : ``to rain'', ``to lie'', ``to rise''}.

\hypertarget{bon-uxe0-savoir-1}{%
\subsection{Bon à savoir}\label{bon-uxe0-savoir-1}}

\hypertarget{une-confusion-fruxe9quente}{%
\subsubsection{Une confusion fréquente}\label{une-confusion-fruxe9quente}}

Ne confondez plus :

\begin{itemize}
\tightlist
\item
  ``\emph{rise (rose risen)}'' et ``\emph{raise}'' ;
\item
  ``\emph{lie (lay lain)}'' et ``\emph{lay}''.
\end{itemize}

\begin{longtable}[]{@{}cc@{}}
\toprule
\begin{minipage}[b]{0.48\columnwidth}\centering
\textbf{Intransitif}\strut
\end{minipage} & \begin{minipage}[b]{0.46\columnwidth}\centering
\textbf{Transitif}\strut
\end{minipage}\tabularnewline
\midrule
\endhead
\begin{minipage}[t]{0.48\columnwidth}\centering
\emph{to rise (rose risen)} \(\rightarrow\) ``élever''\strut
\end{minipage} & \begin{minipage}[t]{0.46\columnwidth}\centering
\emph{to raise stg} \(\rightarrow\) ``lever qqch''\strut
\end{minipage}\tabularnewline
\begin{minipage}[t]{0.48\columnwidth}\centering
\emph{to lie (lay lain)} \(\rightarrow\) ``se situer''\strut
\end{minipage} & \begin{minipage}[t]{0.46\columnwidth}\centering
\emph{to lay stg} \(\rightarrow\) ``poser qqch''\strut
\end{minipage}\tabularnewline
\bottomrule
\end{longtable}

Notez aussi que les deux intransitifs, ``\emph{rise}'' et ``\emph{lie}'', se prononcent avec / /\textipa{eI}/ /\textipa{aI}/ / comme
dans ``\emph{high}''.

Les deux transitifs se prononcent eux avec / /\textipa{i:}/ /\textipa{aI}/ /, comme dans ``\emph{say}''.

\hypertarget{les-adverbes}{%
\section{Les adverbes}\label{les-adverbes}}

Les adverbes fonctionnent comme en français : ils qualifient généralement un verbe et son complément de la même manière qu'un adjectif qualifie un nom.

Attention à l'emplacement des adverbes dits ``de fréquence'', qui s'insèrent souvent entre le sujet et le verbe conjugué :

\begin{itemize}
\tightlist
\item
  \emph{He \textbf{often} speaks to himself} \(\rightarrow\) Il se parle souvent à lui-même.
\end{itemize}

\hypertarget{les-pruxe9positions}{%
\section{Les prépositions}\label{les-pruxe9positions}}

(W.I.P.)

\hypertarget{vocabulaire}{%
\section{Vocabulaire}\label{vocabulaire}}

\hypertarget{dn1}{%
\subsection{DN1}\label{dn1}}

\begin{verbatim}
## Warning in styling_latex_scale_down(out, table_info): Longtable cannot be
## resized.
\end{verbatim}

\begin{longtable}{ll}
\toprule
Français & English\\
\midrule
\cellcolor{gray!6}{15 sur 20} & \cellcolor{gray!6}{15 out of 20}\\

adhérer à qqch & to subscribe to stg\\

\cellcolor{gray!6}{alors que} & \cellcolor{gray!6}{whereas}\\

appartenir à & to belong to\\

\cellcolor{gray!6}{approvisionner, fournir} & \cellcolor{gray!6}{to supply}\\

attendre de qn qu'il fasse qqch & to expect sb to do stg\\

\cellcolor{gray!6}{bien que} & \cellcolor{gray!6}{although}\\

capacité d'attention & attention span\\

\cellcolor{gray!6}{cautionner} & \cellcolor{gray!6}{to endorse}\\

concret & hands-on\\

\cellcolor{gray!6}{de la boue} & \cellcolor{gray!6}{mud}\\

des aiguilles à tricoter & knitting needles\\

\cellcolor{gray!6}{écouter qqch} & \cellcolor{gray!6}{to listen to stg}\\

expliciter, détailler & to spell out\\

\cellcolor{gray!6}{fonder} & \cellcolor{gray!6}{to found}\\

froncer les sourcils, désapprouver qqch & to frown (on stg)\\

\cellcolor{gray!6}{grandir} & \cellcolor{gray!6}{to grow up}\\

interdire (b...) & to ban\\

\cellcolor{gray!6}{interdire (f...)} & \cellcolor{gray!6}{to forbid}\\

interdire (p...) & to prohibit\\

\cellcolor{gray!6}{le siège d'une entreprise} & \cellcolor{gray!6}{the headquarters}\\

mais, pourtant & yet\\

\cellcolor{gray!6}{même si (concession)} & \cellcolor{gray!6}{even though}\\

mettre en oeuvre & to implement\\

\cellcolor{gray!6}{pendant que} & \cellcolor{gray!6}{while}\\

réclamer, exiger & to call for\\

\cellcolor{gray!6}{s'avérer} & \cellcolor{gray!6}{to turn out}\\

se concentrer sur qqch & to focus on stg\\

\cellcolor{gray!6}{s'empresser de} & \cellcolor{gray!6}{to rush to do stg}\\

stable & steady\\

\cellcolor{gray!6}{une politique, une mesure} & \cellcolor{gray!6}{a policy}\\

un intrus & an odd-one-out\\

\cellcolor{gray!6}{un noyau} & \cellcolor{gray!6}{a core}\\

un ordinateur de bureau & a desktop\\

\cellcolor{gray!6}{un ordinateur portable} & \cellcolor{gray!6}{a laptop}\\

un outil & a tool\\

\cellcolor{gray!6}{un pilier} & \cellcolor{gray!6}{a pillar}\\

un portail & a gate\\

\cellcolor{gray!6}{un résumé} & \cellcolor{gray!6}{a summary}\\

un sondage & a poll\\
\bottomrule
\end{longtable}

\hypertarget{dn2}{%
\subsection{DN2}\label{dn2}}

\begin{verbatim}
## Warning in styling_latex_scale_down(out, table_info): Longtable cannot be
## resized.
\end{verbatim}

\begin{longtable}{ll}
\toprule
Français & English\\
\midrule
\cellcolor{gray!6}{aborder (un sujet)} & \cellcolor{gray!6}{to tackle}\\

à haute criminalité & crime-ridden\\

\cellcolor{gray!6}{attendre de qn que...} & \cellcolor{gray!6}{to expect sb to}\\

biaisé, partial & biassed\\

\cellcolor{gray!6}{chaos} & \cellcolor{gray!6}{mayhem}\\

comestible & edible\\

\cellcolor{gray!6}{destinataire, récipiendaire} & \cellcolor{gray!6}{a recipient}\\

diffuser & to broadcast\\

\cellcolor{gray!6}{disposer, agencer} & \cellcolor{gray!6}{to lay out}\\

durer & to last\\

\cellcolor{gray!6}{échapper à} & \cellcolor{gray!6}{to elude}\\

esquisser & to sketch\\

\cellcolor{gray!6}{être susceptible de} & \cellcolor{gray!6}{to be likely to}\\

évident & obvious\\

\cellcolor{gray!6}{exact, précis} & \cellcolor{gray!6}{accurate}\\

fournir (p...) & to provide sb with\\

\cellcolor{gray!6}{fournir (s...)} & \cellcolor{gray!6}{to supply sb with}\\

incliner & to tilt\\

\cellcolor{gray!6}{intenter un procès} & \cellcolor{gray!6}{to sue}\\

interdire & to ban\\

\cellcolor{gray!6}{la majorité écrasante} & \cellcolor{gray!6}{the overwhelming majority}\\

le confinement & the lockdown\\

\cellcolor{gray!6}{menacer} & \cellcolor{gray!6}{to threaten}\\

pendant que & while\\

\cellcolor{gray!6}{posséder qqch} & \cellcolor{gray!6}{to own stg}\\

postuler à qqch & to apply for stg\\

\cellcolor{gray!6}{promouvoir (a...)} & \cellcolor{gray!6}{to advertise}\\

promouvoir (p...) & to promote\\

\cellcolor{gray!6}{provenir de} & \cellcolor{gray!6}{to stem from}\\

résoudre & to work out\\

\cellcolor{gray!6}{s'accroupir} & \cellcolor{gray!6}{to croush}\\

selon, d'après & according to\\

\cellcolor{gray!6}{(se) terminer} & \cellcolor{gray!6}{to be over}\\

sonder & to probe\\

\cellcolor{gray!6}{un but, un objectif (a...)} & \cellcolor{gray!6}{an aim}\\

un but, un objectif (p...) & a purpose\\

\cellcolor{gray!6}{un écart (d...)} & \cellcolor{gray!6}{a discrepancy}\\

un écart (g...) & a gap\\

\cellcolor{gray!6}{une légende d'image} & \cellcolor{gray!6}{a caption}\\

une mission, une tâche & an assignment\\

\cellcolor{gray!6}{une pièce de 25 ¢} & \cellcolor{gray!6}{a quarter}\\

une pièce de 5 ¢ & a nickel\\

\cellcolor{gray!6}{une pièce de 10 ¢} & \cellcolor{gray!6}{a dime}\\

une tâche de couleur & a patch\\

\cellcolor{gray!6}{une vue d'ensemble} & \cellcolor{gray!6}{an overview}\\

un hommage & a tribute\\

\cellcolor{gray!6}{un indice} & \cellcolor{gray!6}{a clue}\\

un ordinateur de bureau & a desktop\\

\cellcolor{gray!6}{un ordinateur portable} & \cellcolor{gray!6}{a laptop}\\

un rédacteur en chef & an editor\\

\cellcolor{gray!6}{un sondage} & \cellcolor{gray!6}{a poll}\\

un stage & an internship\\

\cellcolor{gray!6}{vendre la mèche} & \cellcolor{gray!6}{to spill the beans}\\
\bottomrule
\end{longtable}

\hypertarget{dn3}{%
\subsection{DN3}\label{dn3}}

\begin{verbatim}
## Warning in styling_latex_scale_down(out, table_info): Longtable cannot be
## resized.
\end{verbatim}

\begin{longtable}{ll}
\toprule
Français & English\\
\midrule
\cellcolor{gray!6}{aborder un problème} & \cellcolor{gray!6}{to address an issue}\\

augmenter & to increase\\

\cellcolor{gray!6}{bien que} & \cellcolor{gray!6}{although}\\

dans quelle mesure & to what extent\\

\cellcolor{gray!6}{de plus} & \cellcolor{gray!6}{furthermore}\\

dès le départ & right off the bat\\

\cellcolor{gray!6}{en dépit de} & \cellcolor{gray!6}{despite}\\

exécution, mise en œuvre & implementation\\

\cellcolor{gray!6}{les résultats} & \cellcolor{gray!6}{the findings}\\

mettre à nu & to lay bare\\

\cellcolor{gray!6}{pertinent} & \cellcolor{gray!6}{relevant}\\

puisque & since\\

\cellcolor{gray!6}{se concentrer sur} & \cellcolor{gray!6}{to focus on}\\

un but (p...) & a purpose\\

\cellcolor{gray!6}{un échantillon} & \cellcolor{gray!6}{a sample}\\

une tendance & a trend\\

\cellcolor{gray!6}{un moyen de} & \cellcolor{gray!6}{a means to}\\

un résultat (o...) & an outcome\\

\cellcolor{gray!6}{un résumé} & \cellcolor{gray!6}{a summary}\\

viser à, avoir pour but de & to aim to\\
\bottomrule
\end{longtable}

\hypertarget{erpc-1uxe8re-annuxe9e}{%
\subsection{ERPC 1ère année}\label{erpc-1uxe8re-annuxe9e}}

\begin{verbatim}
## Warning in styling_latex_scale_down(out, table_info): Longtable cannot be
## resized.
\end{verbatim}

\begin{longtable}{ll}
\toprule
Français & English\\
\midrule
\cellcolor{gray!6}{alimenter} & \cellcolor{gray!6}{to feed (fed, fed)}\\

à travers quelque chose & through\\

\cellcolor{gray!6}{autocollants} & \cellcolor{gray!6}{stickers}\\

brochures & booklets\\

\cellcolor{gray!6}{bye} & \cellcolor{gray!6}{au revoir}\\

cartes de visite & business cards\\

\cellcolor{gray!6}{charger} & \cellcolor{gray!6}{to load}\\

commencer & to begin (began, begun)\\

\cellcolor{gray!6}{de la poudre} & \cellcolor{gray!6}{powder}\\

de l'encre & ink\\

\cellcolor{gray!6}{de l'huile} & \cellcolor{gray!6}{oil}\\

dépliants & brochures\\

\cellcolor{gray!6}{des agrafes} & \cellcolor{gray!6}{staples}\\

deux fois & twice\\

\cellcolor{gray!6}{dos carré-collé} & \cellcolor{gray!6}{perfect binding}\\

équipement, installation & facility\\

\cellcolor{gray!6}{étape} & \cellcolor{gray!6}{a stage}\\

expédier & to ship out\\

\cellcolor{gray!6}{fonctionner} & \cellcolor{gray!6}{to work}\\

glisser & to glide\\

\cellcolor{gray!6}{gravé au laser} & \cellcolor{gray!6}{laser-etched}\\

hello & bonjour\\

\cellcolor{gray!6}{humecter, humidifier} & \cellcolor{gray!6}{to dampen}\\

imprimer & to print\\

\cellcolor{gray!6}{item fonctionner} & \cellcolor{gray!6}{to work}\\

le dos (d'un livre) & the spine\\

\cellcolor{gray!6}{le grammage} & \cellcolor{gray!6}{paper weight}\\

le recto & the front\\

\cellcolor{gray!6}{le verso} & \cellcolor{gray!6}{the back}\\

livrer & to deliver\\

\cellcolor{gray!6}{mieux convenir à} & \cellcolor{gray!6}{to be best suited for}\\

mince, fin & thin\\

\cellcolor{gray!6}{pailleté, miroitant} & \cellcolor{gray!6}{shimmery}\\

piqûre à cheval & saddle-stitched\\

\cellcolor{gray!6}{précis, aiguisé} & \cellcolor{gray!6}{sharp}\\

recto-verso & both sides\\

\cellcolor{gray!6}{relier (un livre)} & \cellcolor{gray!6}{to bind (bound, bound)}\\

reliure à spirale & coil binding\\

\cellcolor{gray!6}{résulter dans, aboutir à} & \cellcolor{gray!6}{to result in}\\

sans & without\\

\cellcolor{gray!6}{stokage} & \cellcolor{gray!6}{storage}\\

taille & size\\

\cellcolor{gray!6}{tomber, chuter} & \cellcolor{gray!6}{to fall (fell, fallen)}\\

trade & commerce\\

\cellcolor{gray!6}{une agence de communication} & \cellcolor{gray!6}{an ad(vertising) agency}\\

une configuration & a setup\\

\cellcolor{gray!6}{une couche} & \cellcolor{gray!6}{a layer}\\

une couverture, un blanchet & a blanket\\

\cellcolor{gray!6}{une encoche} & \cellcolor{gray!6}{a notch}\\

une enveloppe & a wrap\\

\cellcolor{gray!6}{une fente} & \cellcolor{gray!6}{a slit}\\

une finition brillante & a glossy finish\\

\cellcolor{gray!6}{une finition mate} & \cellcolor{gray!6}{a matte finish}\\

une fois que & once\\

\cellcolor{gray!6}{une plaque} & \cellcolor{gray!6}{a plate}\\

une sous-couche & an under-coat\\

\cellcolor{gray!6}{une surface, une tache} & \cellcolor{gray!6}{a spot}\\

un exemplaire & a copy\\

\cellcolor{gray!6}{un fichier numérique} & \cellcolor{gray!6}{a digital file}\\

un massicot & a trimmer\\

\cellcolor{gray!6}{un pli} & \cellcolor{gray!6}{a fold}\\

un rouleau & a roller\\

\cellcolor{gray!6}{un sondage} & \cellcolor{gray!6}{a poll}\\

un stage & an internship\\
\bottomrule
\end{longtable}

\hypertarget{erpc-2uxe8me-annuxe9e}{%
\subsection{ERPC 2ème année}\label{erpc-2uxe8me-annuxe9e}}

\begin{verbatim}
## Warning in styling_latex_scale_down(out, table_info): Longtable cannot be
## resized.
\end{verbatim}

\begin{longtable}{ll}
\toprule
Français & \vphantom{1}English\\

\midrule
\cellcolor{gray!6}{aimer faire qqch} & \cellcolor{gray!6}{to like doing stg}\\

ajouter & to add\\

\cellcolor{gray!6}{assister à une réunion} & \cellcolor{gray!6}{to attend a meeting}\\

attentionné & caring\\

\cellcolor{gray!6}{avoir peur de qqch} & \cellcolor{gray!6}{to be afraid of stg}\\

bavard & chatty\\

\cellcolor{gray!6}{bien aimer faire qqch} & \cellcolor{gray!6}{to enjoy doing stg}\\

bien conçu & well-designed\\

\cellcolor{gray!6}{bien s'adapter} & \cellcolor{gray!6}{to fit}\\

blanchir & to bleach\\

\cellcolor{gray!6}{brillant} & \cellcolor{gray!6}{glossy}\\

carton & cardboard (U)\\

\cellcolor{gray!6}{choisir} & \cellcolor{gray!6}{to choose}\\

commander qqch & to order stg\\

\cellcolor{gray!6}{d'apparence professionnelle} & \cellcolor{gray!6}{professional-looking}\\

décrire & to describe\\

\cellcolor{gray!6}{de la cire} & \cellcolor{gray!6}{wax (U)}\\

de la colle & glue (U)\\

\cellcolor{gray!6}{délavé} & \cellcolor{gray!6}{washed out}\\

dépenser (ou passer du temps) & to spend\\

\cellcolor{gray!6}{dorure à chaud} & \cellcolor{gray!6}{hot foil stamping}\\

dos carré-collé & perfect-binding\\

\cellcolor{gray!6}{écorce} & \cellcolor{gray!6}{bark (U)}\\

empiler & to stack\\

\cellcolor{gray!6}{encre} & \cellcolor{gray!6}{ink}\\

enfance & childhood\\

\cellcolor{gray!6}{ \vphantom{7}& }\\

\multirow[t]{-2}{*}{\raggedright\arraybackslash enlever} & \multirow[t]{-2}{*}{\raggedright\arraybackslash to remove}\\

\cellcolor{gray!6}{épaisseur} & \cellcolor{gray!6}{thickness}\\

essayer de faire qqch & to try to do stg\\

\cellcolor{gray!6}{être bon dans qqch} & \cellcolor{gray!6}{to be good  at stg}\\

expédier qqch & to ship out stg\\

\cellcolor{gray!6}{fabriquer qqch} & \cellcolor{gray!6}{to manufacture stg}\\

faire du télétravail & to work from home\\

\cellcolor{gray!6}{Français} & \cellcolor{gray!6}{English}\\
glaçage & glazing\\

\cellcolor{gray!6}{hauteur} & \cellcolor{gray!6}{height}\\

la couverture & the cover\\

\cellcolor{gray!6}{la graisse (typographie)} & \cellcolor{gray!6}{the weight}\\

l'amidon & starch (U)\\

\cellcolor{gray!6}{largeur} & \cellcolor{gray!6}{width}\\

la rogne & cut-offs\\

\cellcolor{gray!6}{le dos d'un livre} & \cellcolor{gray!6}{the spine}\\

longueur & length\\

\cellcolor{gray!6}{lycée} & \cellcolor{gray!6}{high-school}\\

mat & matte\\

\cellcolor{gray!6}{} & \cellcolor{gray!6}{to blend}\\

\multirow[t]{-2}{*}{\raggedright\arraybackslash mélanger} & to mix\\

\cellcolor{gray!6}{obsolète} & \cellcolor{gray!6}{outdated}\\

ondulé (carton) & corrugated\\

\cellcolor{gray!6}{ondulé} & \cellcolor{gray!6}{wavy}\\

ouvert d'esprit & open-minded\\

\cellcolor{gray!6}{paresseux} & \cellcolor{gray!6}{lazy}\\

pelliculage & lamination\\

\cellcolor{gray!6}{permettre à qn de faire qqch} & \cellcolor{gray!6}{to enable sb to do stg}\\

permettre à quelqu'un de faire qqch & to allow sb to do stg\\

\cellcolor{gray!6}{piqûre à cheval} & \cellcolor{gray!6}{saddle-stitching}\\

 \vphantom{6}& \\

\cellcolor{gray!6}{\multirow[t]{-2}{*}{\raggedright\arraybackslash précédent}} & \cellcolor{gray!6}{\multirow[t]{-2}{*}{\raggedright\arraybackslash previous}}\\

profondeur & depth\\

\cellcolor{gray!6}{relier un livre} & \cellcolor{gray!6}{to bind a book}\\

reliure spirales & coil binding\\

\cellcolor{gray!6}{remarquer} & \cellcolor{gray!6}{to notice}\\

réparer & to fix\\

\cellcolor{gray!6}{résumer} & \cellcolor{gray!6}{to summarize}\\

rugueux & rough\\

\cellcolor{gray!6}{sécher} & \cellcolor{gray!6}{to dry}\\

se débarrasser de qqch & to get rid of stg\\

\cellcolor{gray!6}{s'intéresser à qqch} & \cellcolor{gray!6}{to be interested in stg}\\

suivant & next\\

\cellcolor{gray!6}{tard/en retard} & \cellcolor{gray!6}{late}\\

télécharger & to download\\

\cellcolor{gray!6}{téléverser} & \cellcolor{gray!6}{to upload}\\

traiter de & to deal with (dealt x 2)\\

\cellcolor{gray!6}{traiter une commande} & \cellcolor{gray!6}{to run an order}\\

travailleur & hard-working\\

\cellcolor{gray!6}{un appareil électronique} & \cellcolor{gray!6}{a device}\\

un autocollant & a sticker\\

\cellcolor{gray!6}{un bâtonnet (yeux)} & \cellcolor{gray!6}{a rod}\\

un blanchet & un blanket\\

\cellcolor{gray!6}{un bobine} & \cellcolor{gray!6}{a reel}\\

 \vphantom{5}& \\

\cellcolor{gray!6}{\multirow[t]{-2}{*}{\raggedright\arraybackslash un calage}} & \cellcolor{gray!6}{\multirow[t]{-2}{*}{\raggedright\arraybackslash a make-ready}}\\

 \vphantom{4}& \\

\cellcolor{gray!6}{\multirow[t]{-2}{*}{\raggedright\arraybackslash un client}} & \cellcolor{gray!6}{\multirow[t]{-2}{*}{\raggedright\arraybackslash a customer}}\\

un dépliant & a folded leaflet\\

\cellcolor{gray!6}{un devis} & \cellcolor{gray!6}{a quote}\\

une caractéristique, une spécification & a feature\\

\cellcolor{gray!6}{une cellule} & \cellcolor{gray!6}{a cell}\\

une couche & a layer\\

\cellcolor{gray!6}{un écran (d...)} & \cellcolor{gray!6}{a display}\\

un écran (m...) & a monitor\\

\cellcolor{gray!6}{une encoche} & \cellcolor{gray!6}{a notch}\\

 \vphantom{3}& \\

\cellcolor{gray!6}{\multirow[t]{-2}{*}{\raggedright\arraybackslash une entreprise}} & \cellcolor{gray!6}{\multirow[t]{-2}{*}{\raggedright\arraybackslash a company}}\\

une étape & a step\\

\cellcolor{gray!6}{une fente} & \cellcolor{gray!6}{a slit}\\

une forme & a shape\\

\cellcolor{gray!6}{une lame} & \cellcolor{gray!6}{a blade}\\

une livraison & a delivery\\

\cellcolor{gray!6}{une machine empileuse} & \cellcolor{gray!6}{a stacker}\\

un entrepôt & a warehouse\\

\cellcolor{gray!6}{une nuance de couleur} & \cellcolor{gray!6}{a hue}\\

une page & a sheet\\

\cellcolor{gray!6}{une plaque} & \cellcolor{gray!6}{a plate}\\

une poignée & a handle\\

\cellcolor{gray!6}{une police (de caractères)} & \cellcolor{gray!6}{a font}\\

un équilibre & a balance\\

\cellcolor{gray!6}{une rainure} & \cellcolor{gray!6}{a scoring line}\\

une récompense, un prix & a prize\\

\cellcolor{gray!6}{une usine} & \cellcolor{gray!6}{a factory}\\

un logiciel & a software\\

\cellcolor{gray!6}{un manchon} & \cellcolor{gray!6}{a shrinkable sleeve}\\

 \vphantom{2}& \\

\cellcolor{gray!6}{\multirow[t]{-2}{*}{\raggedright\arraybackslash un massicot}} & \cellcolor{gray!6}{\multirow[t]{-2}{*}{\raggedright\arraybackslash a trimmer}}\\

un métier, commerce & a trade\\

\cellcolor{gray!6}{un niveau} & \cellcolor{gray!6}{a level}\\

un nuancier & a fan deck\\

\cellcolor{gray!6}{un plateau} & \cellcolor{gray!6}{a tray}\\

un pli & a fold\\

\cellcolor{gray!6}{un pli-fenêtre} & \cellcolor{gray!6}{a gate-fold}\\

un pli roulé (UK) & a roll fold\\

\cellcolor{gray!6}{un pli roulé (US)} & \cellcolor{gray!6}{a tri/letter fold}\\

un rabat & a flap\\

\cellcolor{gray!6}{un revêtement} & \cellcolor{gray!6}{a coating}\\

 \vphantom{1}& \\

\cellcolor{gray!6}{\multirow[t]{-2}{*}{\raggedright\arraybackslash un rouleau}} & \cellcolor{gray!6}{\multirow[t]{-2}{*}{\raggedright\arraybackslash a roller}}\\

 & \\

\cellcolor{gray!6}{\multirow[t]{-2}{*}{\raggedright\arraybackslash un stage}} & \cellcolor{gray!6}{\multirow[t]{-2}{*}{\raggedright\arraybackslash an internship}}\\

un trait & a stroke\\

\cellcolor{gray!6}{un volume} & \cellcolor{gray!6}{a form}\\

vernis sélectif & spot varnish\\

\cellcolor{gray!6}{vif (couleur)} & \cellcolor{gray!6}{bright}\\

PAO & Desktop Publishing (DTP)\\
\bottomrule
\end{longtable}

\end{document}
