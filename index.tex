% Options for packages loaded elsewhere
\PassOptionsToPackage{unicode}{hyperref}
\PassOptionsToPackage{hyphens}{url}
%
\documentclass[
  10pt,
]{article}
\usepackage{lmodern}
\usepackage{amssymb,amsmath}
\usepackage{ifxetex,ifluatex}
\ifnum 0\ifxetex 1\fi\ifluatex 1\fi=0 % if pdftex
  \usepackage[T1]{fontenc}
  \usepackage[utf8]{inputenc}
  \usepackage{textcomp} % provide euro and other symbols
\else % if luatex or xetex
  \usepackage{unicode-math}
  \defaultfontfeatures{Scale=MatchLowercase}
  \defaultfontfeatures[\rmfamily]{Ligatures=TeX,Scale=1}
\fi
% Use upquote if available, for straight quotes in verbatim environments
\IfFileExists{upquote.sty}{\usepackage{upquote}}{}
\IfFileExists{microtype.sty}{% use microtype if available
  \usepackage[]{microtype}
  \UseMicrotypeSet[protrusion]{basicmath} % disable protrusion for tt fonts
}{}
\makeatletter
\@ifundefined{KOMAClassName}{% if non-KOMA class
  \IfFileExists{parskip.sty}{%
    \usepackage{parskip}
  }{% else
    \setlength{\parindent}{0pt}
    \setlength{\parskip}{6pt plus 2pt minus 1pt}}
}{% if KOMA class
  \KOMAoptions{parskip=half}}
\makeatother
\usepackage{xcolor}
\IfFileExists{xurl.sty}{\usepackage{xurl}}{} % add URL line breaks if available
\IfFileExists{bookmark.sty}{\usepackage{bookmark}}{\usepackage{hyperref}}
\hypersetup{
  pdftitle={Le vocabulaire},
  pdfauthor={Adrien Méli},
  hidelinks,
  pdfcreator={LaTeX via pandoc}}
\urlstyle{same} % disable monospaced font for URLs
\usepackage[margin=1in]{geometry}
\usepackage{longtable,booktabs}
% Correct order of tables after \paragraph or \subparagraph
\usepackage{etoolbox}
\makeatletter
\patchcmd\longtable{\par}{\if@noskipsec\mbox{}\fi\par}{}{}
\makeatother
% Allow footnotes in longtable head/foot
\IfFileExists{footnotehyper.sty}{\usepackage{footnotehyper}}{\usepackage{footnote}}
\makesavenoteenv{longtable}
\usepackage{graphicx}
\makeatletter
\def\maxwidth{\ifdim\Gin@nat@width>\linewidth\linewidth\else\Gin@nat@width\fi}
\def\maxheight{\ifdim\Gin@nat@height>\textheight\textheight\else\Gin@nat@height\fi}
\makeatother
% Scale images if necessary, so that they will not overflow the page
% margins by default, and it is still possible to overwrite the defaults
% using explicit options in \includegraphics[width, height, ...]{}
\setkeys{Gin}{width=\maxwidth,height=\maxheight,keepaspectratio}
% Set default figure placement to htbp
\makeatletter
\def\fps@figure{htbp}
\makeatother
\setlength{\emergencystretch}{3em} % prevent overfull lines
\providecommand{\tightlist}{%
  \setlength{\itemsep}{0pt}\setlength{\parskip}{0pt}}
\setcounter{secnumdepth}{5}
\usepackage{fontawesome}
\usepackage{amsmath}
\usepackage{tipa}
\usepackage{moodle}
\usepackage{hyperref}
\usepackage{booktabs}
\usepackage[utf8]{inputenc}
\usepackage[light,sfdefault]{roboto}
\usepackage{titlesec}
%\usepackage{tufte-book}
%\usepackage{fourier}
%\usepackage{montserrat}
%\usepackage[T1]{fontenc}
%\usepackage[french]{babel}

% -----------------------------------------------------------------
% Hyper Setup
% -----------------------------------------------------------------
\hypersetup{
    %bookmarks=true,         % show bookmarks bar?
    unicode=false,          % non-Latin characters in Acrobat�s bookmarks
    pdftoolbar=true,        % show Acrobat�s toolbar?
    pdfmenubar=true,        % show Acrobat�s menu?
    pdffitwindow=false,     % window fit to page when opened
    pdfstartview={FitH},    % fits the width of the page to the window
    %pdftitle={},    % title
    pdfauthor={Adrien Meli},     % author 
    pdfsubject={Phonological rules},   % subject of the document
    pdfcreator={Creator},   % creator of the document
    pdfproducer={Producer}, % producer of the document
    pdfkeywords={Second Language Acquisition, French, English, phonology}, % list of keywords
    pdfnewwindow=true,     % links in new window
    colorlinks=true,       % false: boxed links; true: colored links
    linkcolor=black,          % color of internal links
    citecolor=black,        % color of links to bibliography
    filecolor=magenta,      % color of file links
    urlcolor=blue,           % color of external links
    bookmarksopen=false,
    anchorcolor=black,
    bookmarksnumbered=true,
    pdfpagemode=UseOutlines,    %None/UseOutlines/UseThumbs/FullScreen
    linktocpage=true
}

% ********************Captions and Hyperreferencing / URL **********************

% Captions: This makes captions of figures use a boldfaced small font.

\usepackage[margin=10pt,font=small,labelfont=bf,labelsep=endash]{caption}

% -----------------------------------------------------------------
% TABLE OF CONTENTS
% -----------------------------------------------------------------
\usepackage[dotinlabels]{titletoc}
\titlecontents{section}[0em] % entries are pushed to the rtight
  {} % code to change the appearance
  {} % section number: increase distance to push to the left
  {\hspace*{3.3em}}
  {\titlerule*[1.9mm]{.}\contentspage}
% remove subsections from TOC
\setcounter{tocdepth}{1}

% multi-line curly brackets

\newenvironment{rightbracedtext}
 {$\kern-\nulldelimiterspace\left.\begin{tabular}{@{}l@{}}}
 {\end{tabular}\right\}$}

\newenvironment{leftbracedtext}{$\left\{\begin{tabular}{@{}l}}{\end{tabular}\right.$}

\titleformat{\chapter}[display]
  {\normalfont\bfseries\filcenter}{\LARGE\thechapter}{1ex}
  {\titlerule[2pt]\vspace{2ex}}[\vspace{1ex}{\titlerule[2pt]}]
\titleformat{name=\chapter,numberless}[display]
  {\normalfont\LARGE\bfseries\filcenter}{}{1ex}
  {\vspace{2ex}}[\vspace{1ex}]

\title{Le vocabulaire}
\author{Adrien Méli}
\date{June 04, 2020}

\begin{document}
\maketitle

{
\setcounter{tocdepth}{1}
\tableofcontents
}
Comment constituer une liste de vocabulaire anglais ?

Le vocabulaire s'apprend plus efficacement si la liste de mots reflète certaines caractéristiques grammaticales.

Voici pour les principales catégories grammaticales quelques conseils sur les informations à inclure dans votre liste de vocabulaire.
Ces conseils sont assortis de quelques rappels et astuces.

\hypertarget{les-noms-anglais}{%
\section{Les noms anglais}\label{les-noms-anglais}}

En anglais, certains noms peuvent se mettre au pluriel, d'autres ne le peuvent pas.

Les premiers sont dits ``\textbf{comptables}'', les deuxièmes, ``\textbf{incomptables}''.

\hypertarget{les-noms-comptables}{%
\subsection{Les noms comptables}\label{les-noms-comptables}}

Ces noms prennent obligatoirement un article au singulier.
Ils se mettent généralement au pluriel avec le suffixe \textless-(e)s\textgreater.

\begin{quote}
\textbf{Conseil :} Dans votre liste de vocabulaire, faites précéder les noms comptables de l'article indéfini ``\emph{a}''.

\emph{Exemple : ``a tool'' \(\rightarrow\)} ``un outil''.
\end{quote}

\hypertarget{les-noms-incomptables}{%
\subsection{Les noms incomptables}\label{les-noms-incomptables}}

Ces noms ne peuvent pas se mettre au pluriel.

\begin{quote}
\textbf{Conseil :} Dans votre liste de vocabulaire, indiquez les noms incomptables par ``\emph{(U)}'', pour ``\emph{uncountable}''.

\emph{Exemple : ``evidence (U)'' \(\rightarrow\)} ``des preuves''.
\end{quote}

\hypertarget{bon-uxe0-savoir}{%
\subsection{Bon à savoir}\label{bon-uxe0-savoir}}

\hypertarget{mots-courants-incomptables}{%
\subsubsection{Mots courants incomptables}\label{mots-courants-incomptables}}

Attention aux noms suivants, qui ne se mettent pas au pluriel en anglais :

\begin{itemize}
\tightlist
\item
  \emph{information (U)}
\item
  \emph{furniture (U)} (``meuble'')
\item
  \emph{advice (U)} (``conseil'')
\end{itemize}

Pour dire ``un meuble'' ou ``un conseil'', on dira ``\emph{a piece of furniture}'' ou ``\emph{a piece of advice}''.

\hypertarget{singuliers-avec--s}{%
\subsubsection{Singuliers avec \textless-s\textgreater{}}\label{singuliers-avec--s}}

Les mots suivants ont un \textless-s\textgreater{} au singulier :

\begin{itemize}
\tightlist
\item
  \emph{a series} (``une série'')
\item
  \emph{a means} (``un moyen'')
\item
  \emph{a species} (``une espèce'')
\end{itemize}

Attention aussi à \textcolor{red}{*the news*} (``les nouvelles''), qui est \textbf{toujours singulier}.

\hypertarget{pluriels-sans--s}{%
\subsubsection{Pluriels sans \textless-s\textgreater{}}\label{pluriels-sans--s}}

Inversement, il existe des noms qui peuvent ou doivent se conjuguer au pluriel :

\begin{itemize}
\tightlist
\item
  \textcolor{red}{*the police*} est \textbf{toujours pluriel}.
\item
  \emph{staff}, \emph{team} ou \emph{family} peuvent se conjuguer au pluriel.
\end{itemize}

\hypertarget{les-adjectifs}{%
\section{Les adjectifs}\label{les-adjectifs}}

Les adjectifs anglais sont \textbf{invariables}, et se placent \textbf{avant} le nom.

\begin{itemize}
\tightlist
\item
  ``Des maisons bleues'' \(\rightarrow\) \emph{blue houses}.
\end{itemize}

\begin{quote}
\textbf{Conseil :} Dans votre liste de vocabulaire, indiquez bien la préposition avec laquelle l'adjectif se construit.

\emph{Exemple : ``similar \textbf{to}'' \(\rightarrow\)} ``semblable à''.
\end{quote}

Voici quelques exemples d'adjectifs fréquents se construisant avec des prépositions différentes du français :

\begin{itemize}
\tightlist
\item
  \emph{responsible \textbf{for}} \(\rightarrow\) ``responsable de''
\item
  \emph{dependent \textbf{on}} \(\rightarrow\) ``dépendant de''
\item
  \emph{addicted \textbf{to}} \(\rightarrow\) ``accroc à''
\item
  \emph{interested \textbf{in}} \(\rightarrow\) ``intéressé par''
\item
  \emph{good \textbf{at}} \(\rightarrow\) ``bon en''
\item
  \emph{different \textbf{from}} \(\rightarrow\) ``différent de''
\item
  \emph{worried \textbf{about}} \(\rightarrow\) ``inquiet de''
\item
  \emph{surprised \textbf{at}} \(\rightarrow\) ``surpris par''
\end{itemize}

\hypertarget{les-verbes}{%
\section{Les verbes}\label{les-verbes}}

\begin{quote}
\textbf{Conseil :} Dans votre liste de vocabulaire, faites précéder les verbes de ``\emph{to}'' afin de les distinguer des autres catégories grammaticales.

\emph{Exemple : ``to row'' \(\rightarrow\)} ``ramer''
\end{quote}

Comme les adjectifs, les verbes s'apprennent avec leur \textbf{construction}.

\hypertarget{les-verbes-transitifs}{%
\subsection{Les verbes transitifs}\label{les-verbes-transitifs}}

On distingue les \textbf{verbes transitifs directs}, qui admettent un complément d'objet direct, des \textbf{verbes transitifs indirects}, dont le complément
est séparé du verbe par une préposition.

Considérez les exemples suivants :

\begin{itemize}
\tightlist
\item
  \emph{to listen \textbf{to} stg} \(\rightarrow\) ``écouter qqch''
\item
  \emph{to look \textbf{for} stg} \(\rightarrow\) ``chercher qqch''
\item
  \emph{to look \textbf{at} stg} \(\rightarrow\) ``regarder qqch''
\end{itemize}

Dans ces exemples, le verbe anglais est transitif indirect, tandis que le verbe français est transitif direct.

À l'inverse :

\begin{itemize}
\tightlist
\item
  \emph{to obey sb} \(\rightarrow\) ``obéir \textbf{à} qn''
\end{itemize}

\begin{quote}
\textbf{Conseil :} Dans votre liste de vocabulaire, indiquez la construction du verbe avec ``\emph{stg}'' (``\emph{something}'') ou ``\emph{sb}'' (``\emph{somebody}'').

\emph{Exemple : ``to blame sb for stg'' \(\rightarrow\)} ``reprocher qqch à qn''
\end{quote}

\hypertarget{les-verbes-intransitifs}{%
\subsection{Les verbes intransitifs}\label{les-verbes-intransitifs}}

Ces verbes se contruisent sans complément, et par conséquent ne peuvent se mettre au passif.

\emph{Exemples : ``to rain'', ``to lie'', ``to rise''}.

\hypertarget{bon-uxe0-savoir-1}{%
\subsection{Bon à savoir}\label{bon-uxe0-savoir-1}}

\hypertarget{une-confusion-fruxe9quente}{%
\subsubsection{Une confusion fréquente}\label{une-confusion-fruxe9quente}}

Ne confondez plus :

\begin{itemize}
\tightlist
\item
  ``\emph{rise (rose risen)}'' et ``\emph{raise}'' ;
\item
  ``\emph{lie (lay lain)}'' et ``\emph{lay}''.
\end{itemize}

\begin{longtable}[]{@{}cc@{}}
\toprule
\begin{minipage}[b]{0.48\columnwidth}\centering
\textbf{Intransitif}\strut
\end{minipage} & \begin{minipage}[b]{0.46\columnwidth}\centering
\textbf{Transitif}\strut
\end{minipage}\tabularnewline
\midrule
\endhead
\begin{minipage}[t]{0.48\columnwidth}\centering
\emph{to rise (rose risen)} \(\rightarrow\) ``élever''\strut
\end{minipage} & \begin{minipage}[t]{0.46\columnwidth}\centering
\emph{to raise stg} \(\rightarrow\) ``lever qqch''\strut
\end{minipage}\tabularnewline
\begin{minipage}[t]{0.48\columnwidth}\centering
\emph{to lie (lay lain)} \(\rightarrow\) ``se situer''\strut
\end{minipage} & \begin{minipage}[t]{0.46\columnwidth}\centering
\emph{to lay stg} \(\rightarrow\) ``poser qqch''\strut
\end{minipage}\tabularnewline
\bottomrule
\end{longtable}

Notez aussi que les deux intransitifs, ``\emph{rise}'' et ``\emph{lie}'', se prononcent avec /NA eI aI NA/ comme
dans ``\emph{high}''.

Les deux transitifs se prononcent eux avec /NA i: aI NA/, comme dans ``\emph{say}''.

\hypertarget{les-adverbes}{%
\section{Les adverbes}\label{les-adverbes}}

Les adverbes fonctionnent comme en français : ils qualifient généralement un verbe et son complément de la même manière qu'un adjectif qualifie un nom.

Attention à l'emplacement des adverbes dits ``de fréquence'', qui s'insèrent souvent entre le sujet et le verbe conjugué :

\begin{itemize}
\tightlist
\item
  \emph{He \textbf{often} speaks to himself} \(\rightarrow\) Il se parle souvent à lui-même.
\end{itemize}

\hypertarget{les-pruxe9positions}{%
\section{Les prépositions}\label{les-pruxe9positions}}

(W.I.P.)

\hypertarget{vocabulaire}{%
\section{Vocabulaire}\label{vocabulaire}}

\hypertarget{dnmade1ang2}{%
\subsection{DNMADE1ANG2}\label{dnmade1ang2}}

\begin{verbatim}
## Warning in styling_latex_scale_down(out, table_info): Longtable cannot be
## resized.
\end{verbatim}

\begin{longtable}{ll}
\toprule
Français &  English\\
\midrule
\rowcolor{gray!6}  embêtant & annoying\\

moins connu & lesser-known\\

\rowcolor{gray!6}  remplir & to fulfil\\

un gardien (immeuble) & a caretaker\\

\rowcolor{gray!6}  intelligent & smarter\\

un ennemi & a foe\\

\rowcolor{gray!6}  un effect domino & a knock-on effect\\

la perte & the loss\\

\rowcolor{gray!6}  nourrir & to feed\\

une augmentation & a rise\\

\rowcolor{gray!6}  une menace & a threat\\

câlin, en peluche & cuddly\\

\rowcolor{gray!6}  une sensation de piqûre & stinging\\

injuste & unfair\\

\rowcolor{gray!6}  une espèce & a \vphantom{1} species\\


un gros problème (expression) & a big deal\\

\rowcolor{gray!6}  qui n'attire pas, qui repousse & off-putting\\

une caractéristique & a feature\\

\rowcolor{gray!6}  les affres & the throes\\

sans emploi & jobless\\

\rowcolor{gray!6}  sans & without\\

voleter, battre des ailes & to flap\\

\rowcolor{gray!6}  étonnant & amazing\\

une ondulation & a ripple\\

\rowcolor{gray!6}  une mare & a pond\\

un galet & a pebble\\

\rowcolor{gray!6}  être témoin de qqch & to witness stg\\

une télécommande & a remote\\

\rowcolor{gray!6}  un cadran & a dial\\

la santé & health (U)\\

\rowcolor{gray!6}  la croissance & growth (U)\\

être engagé & to be committed to stg\\

\rowcolor{gray!6}  espace de sauvegarde & storage\\

fondre & to melt\\

\rowcolor{gray!6}  percer un trou & to drill a hole\\

auxiliaire du conditionnel & would\\

\rowcolor{gray!6}  brancher qqch & to plug in stg\\

un tuyau & a pipe\\

\rowcolor{gray!6}  tordre qqch & to bend\\

digne de qqch & worthy of stg\\

\rowcolor{gray!6}  vérité & truth\\

désaccord & disagreement\\

\rowcolor{gray!6}  authentique & genuine\\

dévoiler, découvrir & uncover\\

\rowcolor{gray!6}  se disputer & argue\\

alléchant & tantalizing\\

\rowcolor{gray!6}  de façon non surprenante & unsurprisingly\\

quelque peu & somewhat\\

\rowcolor{gray!6}  bénéficier de & benefit from\\

simple, facile & straightforward\\

\rowcolor{gray!6}  désobéissance civique & civil disobedience\\

libre échange & free trade\\

\rowcolor{gray!6}  en faveur de & in favour of\\

contre & against\\

\rowcolor{gray!6}  un paquet, un ensemble & bundle\\

défier, remettre en cause & challenge\\

\rowcolor{gray!6}  avoir honte de qqch & ashamed of stg\\

obtenir (les droits) & to secure\\

\rowcolor{gray!6}  une entreprise & a \vphantom{1} company\\


un but & purpose\\

\rowcolor{gray!6}  contre-attaquer & to fight back\\

disponible & available\\

\rowcolor{gray!6}  interdit & banned\\

enquêter & to investigate\\

\rowcolor{gray!6}  moins cher & cheaper\\

constructeurs automobiles & automobile manufacturers\\

\rowcolor{gray!6}  le simple fait & the mere fact\\

enfreindre les règles & to break the rules\\

\rowcolor{gray!6}  une légende (image) & a caption\\

endurer, traverser & to go through\\

\rowcolor{gray!6}  s'effondrer & to collapse\\

une entreprise & a company\\
\rowcolor{gray!6}  une dépression & a nervous breakdown\\

une espèce & a species\\
\rowcolor{gray!6}  mortel & deadly\\

un noyau (planète) & a core\\

\rowcolor{gray!6}  chômage & unemployment\\

un taux & a rate\\

\rowcolor{gray!6}  chasser & to hunt\\

une attente & an expectation\\

\rowcolor{gray!6}  un dé & a dice\\

les deux & both\\

\rowcolor{gray!6}  un jeu de mots & a pun, a play-on-words\\

véhiculer un message & to convey a message\\

\rowcolor{gray!6}  2 sur 4 & 2 out of 4\\

flou & blurry\\

\rowcolor{gray!6}  prévenir, avertir & to warn against\\

transformer en & to turn into\\

\rowcolor{gray!6}  l'arrière-plan & the background\\

dépasser (ici) & to push through\\

\rowcolor{gray!6}  à l'aise & at ease\\

être déprimé & to feel down\\

\rowcolor{gray!6}  quelqu'un vous manque & to miss somebody\\

morose & gloomy\\

\rowcolor{gray!6}  sensible & sensitive\\

lutter & to struggle\\

\rowcolor{gray!6}  incroyable & incredible\\

une foule & a crowd\\

\rowcolor{gray!6}  un paysage & a landscape\\

une lentille (photo) & a lens\\

\rowcolor{gray!6}  le confinement & the lockdown\\

une marque & a brand\\

\rowcolor{gray!6}  une forme & a shape\\

esquisser & to sketch\\

\rowcolor{gray!6}  mouler & to mold\\

une sieste & a nap\\

\rowcolor{gray!6}  une frontière & a boundary\\

lourd & heavy\\

\rowcolor{gray!6}  une chemise & a shirt\\

un ver & a worm\\

\rowcolor{gray!6}  aveugle & blind\\

au-dessus de & above\\

\rowcolor{gray!6}  farine & flour\\
\bottomrule
\end{longtable}

\hypertarget{dn2}{%
\subsection{DN2}\label{dn2}}

\begin{verbatim}
## Warning in styling_latex_scale_down(out, table_info): Longtable cannot be
## resized.
\end{verbatim}

\begin{longtable}{ll}
\toprule
Anglais &  Français\\
\midrule
\rowcolor{gray!6}  to take place & avoir lieu\\

in the foreground & au premier plan\\

\rowcolor{gray!6}  to drown & se noyer\\

to be 1m wide & faire 1m de large\\

\rowcolor{gray!6}  to be 1m deep & faire 1m de profondeur\\

to be 1m high & faire 1m de haut\\

\rowcolor{gray!6}  to be 1m long & faire 1m de long\\

to convey a message & véhiculer un message\\

\rowcolor{gray!6}  a shade & une nuance (couleur)\\

an oil-on-canvas & une huile sur toile\\

\rowcolor{gray!6}  it looks as if & on dirait que\\

a storm & une tempête\\

\rowcolor{gray!6}  bottom-left & en bas à gauche\\

top-right & en haut à droite\\

\rowcolor{gray!6}  to make out stg & distinguer qqch\\

a sunset & un coucher de soleil\\
\bottomrule
\end{longtable}

\hypertarget{erpc}{%
\subsection{ERPC}\label{erpc}}

\begin{verbatim}
## Warning in styling_latex_scale_down(out, table_info): Longtable cannot be
## resized.
\end{verbatim}

\begin{longtable}{ll}
\toprule
Français &  English\\
\midrule
\rowcolor{gray!6}  s'intéresser à qqch & to be interested in stg\\

être bon dans qqch & to be good  at stg\\

\rowcolor{gray!6}  aimer faire qqch & to like doing stg\\

bien aimer faire qqch & to enjoy doing stg\\

\rowcolor{gray!6}  dépenser (ou passer du temps) & to spend\\

choisir & to choose\\

\rowcolor{gray!6}  paresseux & lazy\\

bavard & chatty\\

\rowcolor{gray!6}  tard/en retard & late\\

travailleur & hard-working\\

\rowcolor{gray!6}  ouvert d'esprit & open-minded\\

attentionné & caring\\

\rowcolor{gray!6}  un stage & an internship\\

une entreprise & a company\\

\rowcolor{gray!6}  un pli & a fold\\

un massicot & a trimmer\\

\rowcolor{gray!6}  un logiciel & a software\\

un écran (m...) & a monitor\\

\rowcolor{gray!6}  un écran (d...) & a display\\

relier un livre & to bind a book\\

\rowcolor{gray!6}  la couverture & the cover\\

piqûre à cheval & saddle-stitching\\

\rowcolor{gray!6}  dos carré-collé & perfect-binding\\

reliure spirales & coil binding\\

\rowcolor{gray!6}  expédier qqch & to ship out stg\\

un entrepôt & a warehouse\\

\rowcolor{gray!6}  une nuance de couleur & a hue\\

délavé & washed out\\

\rowcolor{gray!6}  vif (couleur) & bright\\

mat & matte\\

\rowcolor{gray!6}  brillant & glossy\\

un revêtement & a coating\\

\rowcolor{gray!6}  une couche & a layer\\

mélanger & to mix\\

\rowcolor{gray!6}  une forme & a shape\\

un volume & a form\\

\rowcolor{gray!6}  ajouter & to add\\

un équilibre & a balance\\

\rowcolor{gray!6}  commander qqch & to order stg\\

enfance & childhood\\

\rowcolor{gray!6}  un bâtonnet (yeux) & a rod\\

une cellule & a cell\\

\rowcolor{gray!6}  fabriquer qqch & to manufacture stg\\

une usine & a factory\\

\rowcolor{gray!6}  écorce & bark (U)\\

glaçage & glazing\\

\rowcolor{gray!6}  un bobine & a reel\\

l'amidon & starch (U)\\

\rowcolor{gray!6}  blanchir & to bleach\\

un client & a customer\\

\rowcolor{gray!6}  se débarrasser de qqch & to get rid of stg\\

carton & cardboard (U)\\

\rowcolor{gray!6}  une lame & a blade\\

de la colle & glue (U)\\

\rowcolor{gray!6}  ondulé & wavy\\

ondulé (carton) & corrugated\\

\rowcolor{gray!6}  largeur & width\\

longueur & length\\

\rowcolor{gray!6}  profondeur & depth\\

hauteur & height\\

\rowcolor{gray!6}  un plateau & a tray\\

une rainure & a scoring line\\

\rowcolor{gray!6}  de la cire & wax (U)\\

la rogne & cut-offs\\

\rowcolor{gray!6}  un rabat & a flap\\

empiler & to stack\\

\rowcolor{gray!6}  épaisseur & thickness\\

une fente & a slit\\

\rowcolor{gray!6}  une encoche & a notch\\

rugueux & rough\\

\rowcolor{gray!6}  un pli-fenêtre & a gate-fold\\

un nuancier & a fan deck\\

\rowcolor{gray!6}  le dos d'un livre & the spine\\

bien s'adapter & to fit\\

\rowcolor{gray!6}  une poignée & a handle\\

un rouleau & a roller\\

\rowcolor{gray!6}  une police (de caractères) & a font\\

un trait & a stroke\\

\rowcolor{gray!6}  obsolète & outdated\\

un autocollant & a sticker\\

\rowcolor{gray!6}  avoir peur de qqch & to be afraid of stg\\

la graisse (typographie) & the weight\\

\rowcolor{gray!6}  d'apparence professionnelle & professional-looking\\

permettre à qn de faire qqch & to enable sb to do stg\\

\rowcolor{gray!6}  remarquer & to notice\\

réparer & to fix\\

\rowcolor{gray!6}  un niveau & a level\\

bien conçu & well-designed\\

\rowcolor{gray!6}  un devis & a quote\\

un pli roulé (UK) & a roll fold\\

\rowcolor{gray!6}  un pli roulé (US) & a tri/letter fold\\

une livraison & a delivery\\

\rowcolor{gray!6}  pelliculage & lamination\\

un dépliant & a folded leaflet\\

\rowcolor{gray!6}  vernis sélectif & spot varnish\\

dorure à chaud & hot foil stamping\\

\rowcolor{gray!6}  enlever & to remove\\

précédent & previous\\

\rowcolor{gray!6}  suivant & next\\
\bottomrule
\end{longtable}

\end{document}
