% Options for packages loaded elsewhere
\PassOptionsToPackage{unicode}{hyperref}
\PassOptionsToPackage{hyphens}{url}
%
\documentclass[
  10pt,
]{article}
\usepackage{lmodern}
\usepackage{amssymb,amsmath}
\usepackage{ifxetex,ifluatex}
\ifnum 0\ifxetex 1\fi\ifluatex 1\fi=0 % if pdftex
  \usepackage[T1]{fontenc}
  \usepackage[utf8]{inputenc}
  \usepackage{textcomp} % provide euro and other symbols
\else % if luatex or xetex
  \usepackage{unicode-math}
  \defaultfontfeatures{Scale=MatchLowercase}
  \defaultfontfeatures[\rmfamily]{Ligatures=TeX,Scale=1}
\fi
% Use upquote if available, for straight quotes in verbatim environments
\IfFileExists{upquote.sty}{\usepackage{upquote}}{}
\IfFileExists{microtype.sty}{% use microtype if available
  \usepackage[]{microtype}
  \UseMicrotypeSet[protrusion]{basicmath} % disable protrusion for tt fonts
}{}
\makeatletter
\@ifundefined{KOMAClassName}{% if non-KOMA class
  \IfFileExists{parskip.sty}{%
    \usepackage{parskip}
  }{% else
    \setlength{\parindent}{0pt}
    \setlength{\parskip}{6pt plus 2pt minus 1pt}}
}{% if KOMA class
  \KOMAoptions{parskip=half}}
\makeatother
\usepackage{xcolor}
\IfFileExists{xurl.sty}{\usepackage{xurl}}{} % add URL line breaks if available
\IfFileExists{bookmark.sty}{\usepackage{bookmark}}{\usepackage{hyperref}}
\hypersetup{
  pdftitle={Le vocabulaire},
  pdfauthor={Adrien Méli},
  hidelinks,
  pdfcreator={LaTeX via pandoc}}
\urlstyle{same} % disable monospaced font for URLs
\usepackage[margin=1in]{geometry}
\usepackage{longtable,booktabs}
% Correct order of tables after \paragraph or \subparagraph
\usepackage{etoolbox}
\makeatletter
\patchcmd\longtable{\par}{\if@noskipsec\mbox{}\fi\par}{}{}
\makeatother
% Allow footnotes in longtable head/foot
\IfFileExists{footnotehyper.sty}{\usepackage{footnotehyper}}{\usepackage{footnote}}
\makesavenoteenv{longtable}
\usepackage{graphicx}
\makeatletter
\def\maxwidth{\ifdim\Gin@nat@width>\linewidth\linewidth\else\Gin@nat@width\fi}
\def\maxheight{\ifdim\Gin@nat@height>\textheight\textheight\else\Gin@nat@height\fi}
\makeatother
% Scale images if necessary, so that they will not overflow the page
% margins by default, and it is still possible to overwrite the defaults
% using explicit options in \includegraphics[width, height, ...]{}
\setkeys{Gin}{width=\maxwidth,height=\maxheight,keepaspectratio}
% Set default figure placement to htbp
\makeatletter
\def\fps@figure{htbp}
\makeatother
\setlength{\emergencystretch}{3em} % prevent overfull lines
\providecommand{\tightlist}{%
  \setlength{\itemsep}{0pt}\setlength{\parskip}{0pt}}
\setcounter{secnumdepth}{5}
\usepackage{fontawesome}
\usepackage{amsmath}
\usepackage{tipa}
\usepackage{moodle}
\usepackage{hyperref}
\usepackage{booktabs}
\usepackage[utf8]{inputenc}
\usepackage[light,sfdefault]{roboto}
\usepackage{titlesec}
\usepackage{multicol}
%\usepackage{tufte-book}
%\usepackage{fourier}
%\usepackage{montserrat}
%\usepackage[T1]{fontenc}
%\usepackage[french]{babel}

%----------------------------------------------------------------------------------------
%	DEFINE COLOURS
%----------------------------------------------------------------------------------------
%\definecolor{darkPrimaryColor}{HTML}{303F9F}
\definecolor{darkPrimaryColor}{HTML}{2c5272}
\definecolor{primaryColor}{HTML}{3F51B5}
\definecolor{lightPrimaryColor}{HTML}{C5CAE9}
\definecolor{textPrimaryColor}{HTML}{FFFFFF}
\definecolor{accentColor}{HTML}{FF5252}
\definecolor{primaryTextColor}{HTML}{212121}
\definecolor{secondaryTextColor}{HTML}{757575}
\definecolor{dividerColor}{HTML}{BDBDBD}

% -----------------------------------------------------------------
% Hyper Setup
% -----------------------------------------------------------------
\hypersetup{
    %bookmarks=true,         % show bookmarks bar?
    unicode=false,          % non-Latin characters in Acrobat�s bookmarks
    pdftoolbar=true,        % show Acrobat�s toolbar?
    pdfmenubar=true,        % show Acrobat�s menu?
    pdffitwindow=false,     % window fit to page when opened
    pdfstartview={FitH},    % fits the width of the page to the window
    %pdftitle={},    % title
    pdfauthor={Adrien Meli},     % author 
    pdfsubject={Phonological rules},   % subject of the document
    pdfcreator={Creator},   % creator of the document
    pdfproducer={Producer}, % producer of the document
    pdfkeywords={Second Language Acquisition, French, English, phonology}, % list of keywords
    pdfnewwindow=true,     % links in new window
    colorlinks=true,       % false: boxed links; true: colored links
    linkcolor=black,          % color of internal links
    citecolor=black,        % color of links to bibliography
    filecolor=magenta,      % color of file links
    urlcolor=blue,           % color of external links
    bookmarksopen=false,
    anchorcolor=black,
    bookmarksnumbered=true,
    pdfpagemode=UseOutlines,    %None/UseOutlines/UseThumbs/FullScreen
    linktocpage=true
}

% ********************Captions and Hyperreferencing / URL **********************

% Captions: This makes captions of figures use a boldfaced small font.

\usepackage[margin=10pt,font=small,labelfont=bf,labelsep=endash]{caption}

% -----------------------------------------------------------------
% TABLE OF CONTENTS
% -----------------------------------------------------------------
\usepackage[dotinlabels]{titletoc}
\titlecontents{section}[0em] % entries are pushed to the rtight
  {} % code to change the appearance
  {} % section number: increase distance to push to the left
  {\hspace*{3.3em}}
  {\titlerule*[1.9mm]{.}\contentspage}
% remove subsections from TOC
\setcounter{tocdepth}{1}

% multi-line curly brackets

\newenvironment{rightbracedtext}
 {$\kern-\nulldelimiterspace\left.\begin{tabular}{@{}l@{}}}
 {\end{tabular}\right\}$}

\newenvironment{leftbracedtext}{$\left\{\begin{tabular}{@{}l}}{\end{tabular}\right.$}

\titleformat{\chapter}[display]
  {\normalfont\bfseries\filcenter}{\LARGE\thechapter}{1ex}
  {\titlerule[2pt]\vspace{2ex}}[\vspace{1ex}{\titlerule[2pt]}]
\titleformat{name=\chapter,numberless}[display]
  {\normalfont\LARGE\bfseries\filcenter}{}{1ex}
  {\vspace{2ex}}[\vspace{1ex}]

%----------------------------------------------------------------------------------------
%	MODIFY SECTION STYLES
%----------------------------------------------------------------------------------------

%\usepackage{titlesec} % Required for modifying sections
%
%%------------------------------------------------
%% Section
%
%\titleformat
%   {\section} % Section type being modified
%   [block] % Shape type, can be: hang, block, display, runin, leftmargin, rightmargin, drop, wrap, frame
%   {\centering\bfseries\color{darkPrimaryColor}} % Format of the whole section
%   {} % Format of the section label
%   {0pt} % Space between the title and label
%   %{\titlerule\newline{\thetitle. }} % Code before the label
%   {\titlerule\newline} % Code before the label
%   [\titlerule]% after code
%
%\titlespacing{\section}
%   {0pt} % left
%   {\baselineskip} % before title
%   {\baselineskip} % Spacing around section titles, the order is: left, before and after
%
%%------------------------------------------------
%% Subsection
%
%\titleformat
%   {\subsection}
%   [block] % Shape type, can be: hang, block, display, runin, leftmargin, rightmargin, drop, wrap, frame
%   {\itshape \color{accentColor}\bfseries}% format
%   %{\thesubsection.}% label
%   {}% label
%   {6pt}% sep btw label and title
%   {}% before code
%   []% after code
% 
%\titlespacing{\subsection}
%   {6pt}% left
%   {\baselineskip} % before title
%   {\baselineskip} % Spacing around section titles, the order is: left, before and after
%
%%------------------------------------------------
%% Subsubsection
%
%\titleformat
%   {\subsubsection}
%   [block] % Shape type, can be: hang, block, display, runin, leftmargin, rightmargin, drop, wrap, frame
%   {\ttfamily \bfseries \itshape }% format
%   {\thesubsubsection}% label
%   {4pt}% sep btw label and title
%   {}% before code
%   []% after code
% 
%\titlespacing{\subsubsection}
%   {12pt}% left
%   {\baselineskip} % before title
%   {\baselineskip}% after sep
%
%%\renewcommand\thesubsection{(\alph{subsection})}
%
%% from bookdown to create column environments
%\newenvironment{columns}[1][]{}{}
%
%\newenvironment{column}[1]{\begin{minipage}{#1}\ignorespaces}{%
%\end{minipage}
%\ifhmode\unskip\fi
%\aftergroup\useignorespacesandallpars}
%
%\def\useignorespacesandallpars#1\ignorespaces\fi{%
%#1\fi\ignorespacesandallpars}
%
%\makeatletter
%\def\ignorespacesandallpars{%
%  \@ifnextchar\par
%    {\expandafter\ignorespacesandallpars\@gobble}%
%    {}%
%}
%\makeatother
\ifluatex
  \usepackage{selnolig}  % disable illegal ligatures
\fi

\title{Le vocabulaire}
\author{Adrien Méli}
\date{October 06, 2020}

\begin{document}
\maketitle

{
\setcounter{tocdepth}{1}
\tableofcontents
}
Comment constituer une liste de vocabulaire anglais ?

Le vocabulaire s'apprend plus efficacement si la liste de mots reflète certaines caractéristiques grammaticales.

Voici pour les principales catégories grammaticales quelques conseils sur les informations à inclure dans votre liste de vocabulaire.
Ces conseils sont assortis de quelques rappels et astuces.

\hypertarget{les-noms-anglais}{%
\section{Les noms anglais}\label{les-noms-anglais}}

En anglais, certains noms peuvent se mettre au pluriel, d'autres ne le peuvent pas.

Les premiers sont dits ``\textbf{comptables}'', les deuxièmes, ``\textbf{incomptables}''.

\hypertarget{les-noms-comptables}{%
\subsection{Les noms comptables}\label{les-noms-comptables}}

Ces noms prennent obligatoirement un article au singulier.
Ils se mettent généralement au pluriel avec le suffixe \textless-(e)s\textgreater.

\begin{quote}
\textbf{Conseil :} Dans votre liste de vocabulaire, faites précéder les noms comptables de l'article indéfini ``\emph{a}''.

\emph{Exemple : ``a tool'' \(\rightarrow\)} ``un outil''.
\end{quote}

\hypertarget{les-noms-incomptables}{%
\subsection{Les noms incomptables}\label{les-noms-incomptables}}

Ces noms ne peuvent pas se mettre au pluriel.

\begin{quote}
\textbf{Conseil :} Dans votre liste de vocabulaire, indiquez les noms incomptables par ``\emph{(U)}'', pour ``\emph{uncountable}''.

\emph{Exemple : ``evidence (U)'' \(\rightarrow\)} ``des preuves''.
\end{quote}

\hypertarget{bon-uxe0-savoir}{%
\subsection{Bon à savoir}\label{bon-uxe0-savoir}}

\hypertarget{mots-courants-incomptables}{%
\subsubsection{Mots courants incomptables}\label{mots-courants-incomptables}}

Attention aux noms suivants, qui ne se mettent pas au pluriel en anglais :

\begin{itemize}
\tightlist
\item
  \emph{information (U)}
\item
  \emph{furniture (U)} (``meuble'')
\item
  \emph{advice (U)} (``conseil'')
\end{itemize}

Pour dire ``un meuble'' ou ``un conseil'', on dira ``\emph{a piece of furniture}'' ou ``\emph{a piece of advice}''.

\hypertarget{singuliers-avec--s}{%
\subsubsection{Singuliers avec \textless-s\textgreater{}}\label{singuliers-avec--s}}

Les mots suivants ont un \textless-s\textgreater{} au singulier :

\begin{itemize}
\tightlist
\item
  \emph{a series} (``une série'')
\item
  \emph{a means} (``un moyen'')
\item
  \emph{a species} (``une espèce'')
\end{itemize}

Attention aussi à \color[HTML]{red}\emph{the news} \color{black} (``les nouvelles''), qui est \textbf{toujours singulier}.

\hypertarget{pluriels-sans--s}{%
\subsubsection{Pluriels sans \textless-s\textgreater{}}\label{pluriels-sans--s}}

Inversement, il existe des noms qui peuvent ou doivent se conjuguer au pluriel :

\begin{itemize}
\tightlist
\item
  \color[HTML]{red}\emph{the police} \color{black} est \textbf{toujours pluriel}.
\item
  \emph{staff}, \emph{team} ou \emph{family} peuvent se conjuguer au pluriel.
\end{itemize}

\hypertarget{les-adjectifs}{%
\section{Les adjectifs}\label{les-adjectifs}}

Les adjectifs anglais sont \textbf{invariables}, et se placent \textbf{avant} le nom.

\begin{itemize}
\tightlist
\item
  ``Des maisons bleues'' \(\rightarrow\) \emph{blue houses}.
\end{itemize}

\begin{quote}
\textbf{Conseil :} Dans votre liste de vocabulaire, indiquez bien la préposition avec laquelle l'adjectif se construit.

\emph{Exemple : ``similar \textbf{to}'' \(\rightarrow\)} ``semblable à''.
\end{quote}

Voici quelques exemples d'adjectifs fréquents se construisant avec des prépositions différentes du français :

\begin{itemize}
\tightlist
\item
  \emph{responsible \textbf{for}} \(\rightarrow\) ``responsable de''
\item
  \emph{dependent \textbf{on}} \(\rightarrow\) ``dépendant de''
\item
  \emph{addicted \textbf{to}} \(\rightarrow\) ``accroc à''
\item
  \emph{interested \textbf{in}} \(\rightarrow\) ``intéressé par''
\item
  \emph{good \textbf{at}} \(\rightarrow\) ``bon en''
\item
  \emph{different \textbf{from}} \(\rightarrow\) ``différent de''
\item
  \emph{worried \textbf{about}} \(\rightarrow\) ``inquiet de''
\item
  \emph{surprised \textbf{at}} \(\rightarrow\) ``surpris par''
\end{itemize}

\hypertarget{les-verbes}{%
\section{Les verbes}\label{les-verbes}}

\begin{quote}
\textbf{Conseil :} Dans votre liste de vocabulaire, faites précéder les verbes de ``\emph{to}'' afin de les distinguer des autres catégories grammaticales.

\emph{Exemple : ``to row'' \(\rightarrow\)} ``ramer''
\end{quote}

Comme les adjectifs, les verbes s'apprennent avec leur \textbf{construction}.

\hypertarget{les-verbes-transitifs}{%
\subsection{Les verbes transitifs}\label{les-verbes-transitifs}}

On distingue les \textbf{verbes transitifs directs}, qui admettent un complément d'objet direct, des \textbf{verbes transitifs indirects}, dont le complément
est séparé du verbe par une préposition.

Considérez les exemples suivants :

\begin{itemize}
\tightlist
\item
  \emph{to listen \textbf{to} stg} \(\rightarrow\) ``écouter qqch''
\item
  \emph{to look \textbf{for} stg} \(\rightarrow\) ``chercher qqch''
\item
  \emph{to look \textbf{at} stg} \(\rightarrow\) ``regarder qqch''
\end{itemize}

Dans ces exemples, le verbe anglais est transitif indirect, tandis que le verbe français est transitif direct.

À l'inverse :

\begin{itemize}
\tightlist
\item
  \emph{to obey sb} \(\rightarrow\) ``obéir \textbf{à} qn''
\end{itemize}

\begin{quote}
\textbf{Conseil :} Dans votre liste de vocabulaire, indiquez la construction du verbe avec ``\emph{stg}'' (``\emph{something}'') ou ``\emph{sb}'' (``\emph{somebody}'').

\emph{Exemple : ``to blame sb for stg'' \(\rightarrow\)} ``reprocher qqch à qn''
\end{quote}

\hypertarget{les-verbes-intransitifs}{%
\subsection{Les verbes intransitifs}\label{les-verbes-intransitifs}}

Ces verbes se contruisent sans complément, et par conséquent ne peuvent se mettre au passif.

\emph{Exemples : ``to rain'', ``to lie'', ``to rise''}.

\hypertarget{bon-uxe0-savoir-1}{%
\subsection{Bon à savoir}\label{bon-uxe0-savoir-1}}

\hypertarget{une-confusion-fruxe9quente}{%
\subsubsection{Une confusion fréquente}\label{une-confusion-fruxe9quente}}

Ne confondez plus :

\begin{itemize}
\tightlist
\item
  ``\emph{rise (rose risen)}'' et ``\emph{raise}'' ;
\item
  ``\emph{lie (lay lain)}'' et ``\emph{lay}''.
\end{itemize}

\begin{longtable}[]{@{}cc@{}}
\toprule
\begin{minipage}[b]{0.48\columnwidth}\centering
\textbf{Intransitif}\strut
\end{minipage} & \begin{minipage}[b]{0.46\columnwidth}\centering
\textbf{Transitif}\strut
\end{minipage}\tabularnewline
\midrule
\endhead
\begin{minipage}[t]{0.48\columnwidth}\centering
\emph{to rise (rose risen)} \(\rightarrow\) ``élever''\strut
\end{minipage} & \begin{minipage}[t]{0.46\columnwidth}\centering
\emph{to raise stg} \(\rightarrow\) ``lever qqch''\strut
\end{minipage}\tabularnewline
\begin{minipage}[t]{0.48\columnwidth}\centering
\emph{to lie (lay lain)} \(\rightarrow\) ``se situer''\strut
\end{minipage} & \begin{minipage}[t]{0.46\columnwidth}\centering
\emph{to lay stg} \(\rightarrow\) ``poser qqch''\strut
\end{minipage}\tabularnewline
\bottomrule
\end{longtable}

Notez aussi que les deux intransitifs, ``\emph{rise}'' et ``\emph{lie}'', se prononcent avec / /\textipa{eI}/ /\textipa{aI}/ / comme
dans ``\emph{high}''.

Les deux transitifs se prononcent eux avec / /\textipa{i:}/ /\textipa{aI}/ /, comme dans ``\emph{say}''.

\hypertarget{les-adverbes}{%
\section{Les adverbes}\label{les-adverbes}}

Les adverbes fonctionnent comme en français : ils qualifient généralement un verbe et son complément de la même manière qu'un adjectif qualifie un nom.

Attention à l'emplacement des adverbes dits ``de fréquence'', qui s'insèrent souvent entre le sujet et le verbe conjugué :

\begin{itemize}
\tightlist
\item
  \emph{He \textbf{often} speaks to himself} \(\rightarrow\) Il se parle souvent à lui-même.
\end{itemize}

\hypertarget{les-pruxe9positions}{%
\section{Les prépositions}\label{les-pruxe9positions}}

(W.I.P.)

\hypertarget{vocabulaire}{%
\section{Vocabulaire}\label{vocabulaire}}

\hypertarget{dn1}{%
\subsection{DN1}\label{dn1}}

\begin{verbatim}
## Warning in styling_latex_scale_down(out, table_info): Longtable cannot be
## resized.
\end{verbatim}

\begin{longtable}{ll}
\toprule
15 sur 20 & 15 out of 20\\
\midrule
\cellcolor{gray!6}{adhérer à qqch} & \cellcolor{gray!6}{to subscribe to stg}\\

alors que & whereas\\

\cellcolor{gray!6}{appartenir à} & \cellcolor{gray!6}{to belong to}\\

approvisionner, fournir & to supply\\

\cellcolor{gray!6}{attendre de qn qu'il fasse qqch} & \cellcolor{gray!6}{to expect sb to do stg}\\

bien que & although\\

\cellcolor{gray!6}{capacité d'attention} & \cellcolor{gray!6}{attention span}\\

cautionner & to endorse\\

\cellcolor{gray!6}{concret} & \cellcolor{gray!6}{hands-on}\\

de la boue & mud\\

\cellcolor{gray!6}{des aiguilles à tricoter} & \cellcolor{gray!6}{knitting needles}\\

écouter qqch & to listen to stg\\

\cellcolor{gray!6}{expliciter, détailler} & \cellcolor{gray!6}{to spell out}\\

fonder & to found\\

\cellcolor{gray!6}{froncer les sourcils, désapprouver qqch} & \cellcolor{gray!6}{to frown (on stg)}\\

grandir & to grow up\\

\cellcolor{gray!6}{interdire (b...)} & \cellcolor{gray!6}{to ban}\\

interdire (f...) & to forbid\\

\cellcolor{gray!6}{interdire (p...)} & \cellcolor{gray!6}{to prohibit}\\

le siège d'une entreprise & the headquarters\\

\cellcolor{gray!6}{mais, pourtant} & \cellcolor{gray!6}{yet}\\

même si (concession) & even though\\

\cellcolor{gray!6}{mettre en oeuvre} & \cellcolor{gray!6}{to implement}\\

pendant que & while\\

\cellcolor{gray!6}{réclamer, exiger} & \cellcolor{gray!6}{to call for}\\

s'avérer & to turn out\\

\cellcolor{gray!6}{se concentrer sur qqch} & \cellcolor{gray!6}{to focus on stg}\\

s'empresser de & to rush to do stg\\

\cellcolor{gray!6}{stable} & \cellcolor{gray!6}{steady}\\

une politique, une mesure & a policy\\

\cellcolor{gray!6}{un intrus} & \cellcolor{gray!6}{an odd-one-out}\\

un noyau & a core\\

\cellcolor{gray!6}{un ordinateur de bureau} & \cellcolor{gray!6}{a desktop}\\

un ordinateur portable & a laptop\\

\cellcolor{gray!6}{un outil} & \cellcolor{gray!6}{a tool}\\

un pilier & a pillar\\

\cellcolor{gray!6}{un portail} & \cellcolor{gray!6}{a gate}\\

un résumé & a summary\\

\cellcolor{gray!6}{un sondage} & \cellcolor{gray!6}{a poll}\\
\bottomrule
\end{longtable}

\hypertarget{dn2}{%
\subsection{DN2}\label{dn2}}

\begin{verbatim}
## Warning in styling_latex_scale_down(out, table_info): Longtable cannot be
## resized.
\end{verbatim}

\begin{longtable}{ll}
\toprule
aborder (un sujet) & to tackle\\
\midrule
\cellcolor{gray!6}{à haute criminalité} & \cellcolor{gray!6}{crime-ridden}\\

attendre de qn que... & to expect sb to\\

\cellcolor{gray!6}{biaisé, partial} & \cellcolor{gray!6}{biassed}\\

chaos & mayhem\\

\cellcolor{gray!6}{comestible} & \cellcolor{gray!6}{edible}\\

destinataire, récipiendaire & a recipient\\

\cellcolor{gray!6}{diffuser} & \cellcolor{gray!6}{to broadcast}\\

disposer, agencer & to lay out\\

\cellcolor{gray!6}{durer} & \cellcolor{gray!6}{to last}\\

échapper à & to elude\\

\cellcolor{gray!6}{esquisser} & \cellcolor{gray!6}{to sketch}\\

être susceptible de & to be likely to\\

\cellcolor{gray!6}{évident} & \cellcolor{gray!6}{obvious}\\

exact, précis & accurate\\

\cellcolor{gray!6}{fournir (p...)} & \cellcolor{gray!6}{to provide sb with}\\

fournir (s...) & to supply sb with\\

\cellcolor{gray!6}{incliner} & \cellcolor{gray!6}{to tilt}\\

intenter un procès & to sue\\

\cellcolor{gray!6}{interdire} & \cellcolor{gray!6}{to ban}\\

la majorité écrasante & the overwhelming majority\\

\cellcolor{gray!6}{le confinement} & \cellcolor{gray!6}{the lockdown}\\

menacer & to threaten\\

\cellcolor{gray!6}{pendant que} & \cellcolor{gray!6}{while}\\

posséder qqch & to own stg\\

\cellcolor{gray!6}{postuler à qqch} & \cellcolor{gray!6}{to apply for stg}\\

promouvoir (a...) & to advertise\\

\cellcolor{gray!6}{promouvoir (p...)} & \cellcolor{gray!6}{to promote}\\

provenir de & to stem from\\

\cellcolor{gray!6}{résoudre} & \cellcolor{gray!6}{to work out}\\

s'accroupir & to croush\\

\cellcolor{gray!6}{selon, d'après} & \cellcolor{gray!6}{according to}\\

(se) terminer & to be over\\

\cellcolor{gray!6}{sonder} & \cellcolor{gray!6}{to probe}\\

un but, un objectif (a...) & an aim\\

\cellcolor{gray!6}{un but, un objectif (p...)} & \cellcolor{gray!6}{a purpose}\\

un écart (d...) & a discrepancy\\

\cellcolor{gray!6}{un écart (g...)} & \cellcolor{gray!6}{a gap}\\

une légende d'image & a caption\\

\cellcolor{gray!6}{une mission, une tâche} & \cellcolor{gray!6}{an assignment}\\

une pièce de 25 ¢ & a quarter\\

\cellcolor{gray!6}{une pièce de 5 ¢} & \cellcolor{gray!6}{a nickel}\\

une pièce de 10 ¢ & a dime\\

\cellcolor{gray!6}{une tâche de couleur} & \cellcolor{gray!6}{a patch}\\

une vue d'ensemble & an overview\\

\cellcolor{gray!6}{un hommage} & \cellcolor{gray!6}{a tribute}\\

un indice & a clue\\

\cellcolor{gray!6}{un ordinateur de bureau} & \cellcolor{gray!6}{a desktop}\\

un ordinateur portable & a laptop\\

\cellcolor{gray!6}{un rédacteur en chef} & \cellcolor{gray!6}{an editor}\\

un sondage & a poll\\

\cellcolor{gray!6}{un stage} & \cellcolor{gray!6}{an internship}\\

vendre la mèche & to spill the beans\\
\bottomrule
\end{longtable}

\hypertarget{dn3}{%
\subsection{DN3}\label{dn3}}

\begin{verbatim}
## Warning in styling_latex_scale_down(out, table_info): Longtable cannot be
## resized.
\end{verbatim}

\begin{longtable}{ll}
\toprule
aborder un problème & to address an issue\\
\midrule
\cellcolor{gray!6}{augmenter} & \cellcolor{gray!6}{to increase}\\

bien que & although\\

\cellcolor{gray!6}{dans quelle mesure} & \cellcolor{gray!6}{to what extent}\\

de plus & furthermore\\

\cellcolor{gray!6}{dès le départ} & \cellcolor{gray!6}{right off the bat}\\

en dépit de & despite\\

\cellcolor{gray!6}{exécution, mise en œuvre} & \cellcolor{gray!6}{implementation}\\

les résultats & the findings\\

\cellcolor{gray!6}{mettre à nu} & \cellcolor{gray!6}{to lay bare}\\

pertinent & relevant\\

\cellcolor{gray!6}{puisque} & \cellcolor{gray!6}{since}\\

se concentrer sur & to focus on\\

\cellcolor{gray!6}{un but (p...)} & \cellcolor{gray!6}{a purpose}\\

un échantillon & a sample\\

\cellcolor{gray!6}{une tendance} & \cellcolor{gray!6}{a trend}\\

un moyen de & a means to\\

\cellcolor{gray!6}{un résultat (o...)} & \cellcolor{gray!6}{an outcome}\\

un résumé & a summary\\

\cellcolor{gray!6}{viser à, avoir pour but de} & \cellcolor{gray!6}{to aim to}\\
\bottomrule
\end{longtable}

\hypertarget{erpc-1uxe8re-annuxe9e}{%
\subsection{ERPC 1ère année}\label{erpc-1uxe8re-annuxe9e}}

\begin{verbatim}
## Warning in styling_latex_scale_down(out, table_info): Longtable cannot be
## resized.
\end{verbatim}

\begin{longtable}{ll}
\toprule
alimenter & to feed (fed, fed)\\
\midrule
\cellcolor{gray!6}{à travers quelque chose} & \cellcolor{gray!6}{through}\\

autocollants & stickers\\

\cellcolor{gray!6}{brochures} & \cellcolor{gray!6}{booklets}\\

bye & au revoir\\

\cellcolor{gray!6}{cartes de visite} & \cellcolor{gray!6}{business cards}\\

charger & to load\\

\cellcolor{gray!6}{commencer} & \cellcolor{gray!6}{to begin (began, begun)}\\

de la poudre & powder\\

\cellcolor{gray!6}{de l'encre} & \cellcolor{gray!6}{ink}\\

de l'huile & oil\\

\cellcolor{gray!6}{dépliants} & \cellcolor{gray!6}{brochures}\\

des agrafes & staples\\

\cellcolor{gray!6}{deux fois} & \cellcolor{gray!6}{twice}\\

dos carré-collé & perfect binding\\

\cellcolor{gray!6}{équipement, installation} & \cellcolor{gray!6}{facility}\\

étape & a stage\\

\cellcolor{gray!6}{expédier} & \cellcolor{gray!6}{to ship out}\\

fonctionner & to work\\

\cellcolor{gray!6}{glisser} & \cellcolor{gray!6}{to glide}\\

gravé au laser & laser-etched\\

\cellcolor{gray!6}{hello} & \cellcolor{gray!6}{bonjour}\\

humecter, humidifier & to dampen\\

\cellcolor{gray!6}{imprimer} & \cellcolor{gray!6}{to print}\\

item fonctionner & to work\\

\cellcolor{gray!6}{le dos (d'un livre)} & \cellcolor{gray!6}{the spine}\\

le grammage & paper weight\\

\cellcolor{gray!6}{le recto} & \cellcolor{gray!6}{the front}\\

le verso & the back\\

\cellcolor{gray!6}{livrer} & \cellcolor{gray!6}{to deliver}\\

mieux convenir à & to be best suited for\\

\cellcolor{gray!6}{mince, fin} & \cellcolor{gray!6}{thin}\\

pailleté, miroitant & shimmery\\

\cellcolor{gray!6}{piqûre à cheval} & \cellcolor{gray!6}{saddle-stitched}\\

précis, aiguisé & sharp\\

\cellcolor{gray!6}{recto-verso} & \cellcolor{gray!6}{both sides}\\

relier (un livre) & to bind (bound, bound)\\

\cellcolor{gray!6}{reliure à spirale} & \cellcolor{gray!6}{coil binding}\\

résulter dans, aboutir à & to result in\\

\cellcolor{gray!6}{sans} & \cellcolor{gray!6}{without}\\

stokage & storage\\

\cellcolor{gray!6}{taille} & \cellcolor{gray!6}{size}\\

tomber, chuter & to fall (fell, fallen)\\

\cellcolor{gray!6}{trade} & \cellcolor{gray!6}{commerce}\\

une agence de communication & an ad(vertising) agency\\

\cellcolor{gray!6}{une configuration} & \cellcolor{gray!6}{a setup}\\

une couche & a layer\\

\cellcolor{gray!6}{une couverture, un blanchet} & \cellcolor{gray!6}{a blanket}\\

une encoche & a notch\\

\cellcolor{gray!6}{une enveloppe} & \cellcolor{gray!6}{a wrap}\\

une fente & a slit\\

\cellcolor{gray!6}{une finition brillante} & \cellcolor{gray!6}{a glossy finish}\\

une finition mate & a matte finish\\

\cellcolor{gray!6}{une fois que} & \cellcolor{gray!6}{once}\\

une plaque & a plate\\

\cellcolor{gray!6}{une sous-couche} & \cellcolor{gray!6}{an under-coat}\\

une surface, une tache & a spot\\

\cellcolor{gray!6}{un exemplaire} & \cellcolor{gray!6}{a copy}\\

un fichier numérique & a digital file\\

\cellcolor{gray!6}{un massicot} & \cellcolor{gray!6}{a trimmer}\\

un pli & a fold\\

\cellcolor{gray!6}{un rouleau} & \cellcolor{gray!6}{a roller}\\

un sondage & a poll\\

\cellcolor{gray!6}{un stage} & \cellcolor{gray!6}{an internship}\\
\bottomrule
\end{longtable}

\hypertarget{erpc-2uxe8me-annuxe9e}{%
\subsection{ERPC 2ème année}\label{erpc-2uxe8me-annuxe9e}}

\begin{verbatim}
## Warning in styling_latex_scale_down(out, table_info): Longtable cannot be
## resized.
\end{verbatim}

\begin{longtable}{ll}
\toprule
aimer faire qqch\cellcolor{gray!6}{} & \cellcolor{gray!6}{}to like doing stg\\
\midrule
\cellcolor{gray!6}{ajouter} & \cellcolor{gray!6}{to add}\\

assister à une réunion & to attend a meeting\\

\cellcolor{gray!6}{attentionné} & \cellcolor{gray!6}{caring}\\

avoir peur de qqch & to be afraid of stg\\

\cellcolor{gray!6}{bavard} & \cellcolor{gray!6}{chatty}\\

bien aimer faire qqch & to enjoy doing stg\\

\cellcolor{gray!6}{bien conçu} & \cellcolor{gray!6}{well-designed}\\

bien s'adapter & to fit\\

\cellcolor{gray!6}{blanchir} & \cellcolor{gray!6}{to bleach}\\

brillant & glossy\\

\cellcolor{gray!6}{carton} & \cellcolor{gray!6}{cardboard (U)}\\

choisir & to choose\\

\cellcolor{gray!6}{commander qqch} & \cellcolor{gray!6}{to order stg}\\

d'apparence professionnelle & professional-looking\\

\cellcolor{gray!6}{décrire} & \cellcolor{gray!6}{to describe}\\

de la cire & wax (U)\\

\cellcolor{gray!6}{de la colle} & \cellcolor{gray!6}{glue (U)}\\

délavé & washed out\\

\cellcolor{gray!6}{dépenser (ou passer du temps)} & \cellcolor{gray!6}{to spend}\\

dorure à chaud & hot foil stamping\\

\cellcolor{gray!6}{dos carré-collé} & \cellcolor{gray!6}{perfect-binding}\\

écorce & bark (U)\\

\cellcolor{gray!6}{empiler} & \cellcolor{gray!6}{to stack}\\

encre & ink\\

\cellcolor{gray!6}{enfance} & \cellcolor{gray!6}{childhood}\\

 \vphantom{7}& \\

\cellcolor{gray!6}{\multirow[t]{-2}{*}{\raggedright\arraybackslash enlever}} & \cellcolor{gray!6}{\multirow[t]{-2}{*}{\raggedright\arraybackslash to remove}}\\

épaisseur & thickness\\

\cellcolor{gray!6}{essayer de faire qqch} & \cellcolor{gray!6}{to try to do stg}\\

être bon dans qqch & to be good  at stg\\

\cellcolor{gray!6}{expédier qqch} & \cellcolor{gray!6}{to ship out stg}\\

fabriquer qqch & to manufacture stg\\

\cellcolor{gray!6}{faire du télétravail} & \cellcolor{gray!6}{to work from home}\\

Français & English\\

\cellcolor{gray!6}{glaçage} & \cellcolor{gray!6}{glazing}\\

hauteur & height\\

\cellcolor{gray!6}{la couverture} & \cellcolor{gray!6}{the cover}\\

la graisse (typographie) & the weight\\

\cellcolor{gray!6}{l'amidon} & \cellcolor{gray!6}{starch (U)}\\

largeur & width\\

\cellcolor{gray!6}{la rogne} & \cellcolor{gray!6}{cut-offs}\\

le dos d'un livre & the spine\\

\cellcolor{gray!6}{longueur} & \cellcolor{gray!6}{length}\\

lycée & high-school\\

\cellcolor{gray!6}{mat} & \cellcolor{gray!6}{matte}\\

 & to blend\\

\cellcolor{gray!6}{\multirow[t]{-2}{*}{\raggedright\arraybackslash mélanger}} & \cellcolor{gray!6}{to mix}\\

obsolète & outdated\\

\cellcolor{gray!6}{ondulé (carton)} & \cellcolor{gray!6}{corrugated}\\

ondulé & wavy\\

\cellcolor{gray!6}{ouvert d'esprit} & \cellcolor{gray!6}{open-minded}\\

paresseux & lazy\\

\cellcolor{gray!6}{pelliculage} & \cellcolor{gray!6}{lamination}\\

permettre à qn de faire qqch & to enable sb to do stg\\

\cellcolor{gray!6}{permettre à quelqu'un de faire qqch} & \cellcolor{gray!6}{to allow sb to do stg}\\

piqûre à cheval & saddle-stitching\\

\cellcolor{gray!6}{ \vphantom{6}& }\\

\multirow[t]{-2}{*}{\raggedright\arraybackslash précédent} & \multirow[t]{-2}{*}{\raggedright\arraybackslash previous}\\

\cellcolor{gray!6}{profondeur} & \cellcolor{gray!6}{depth}\\

relier un livre & to bind a book\\

\cellcolor{gray!6}{reliure spirales} & \cellcolor{gray!6}{coil binding}\\

remarquer & to notice\\

\cellcolor{gray!6}{réparer} & \cellcolor{gray!6}{to fix}\\

résumer & to summarize\\

\cellcolor{gray!6}{rugueux} & \cellcolor{gray!6}{rough}\\

sécher & to dry\\

\cellcolor{gray!6}{se débarrasser de qqch} & \cellcolor{gray!6}{to get rid of stg}\\

s'intéresser à qqch & to be interested in stg\\

\cellcolor{gray!6}{suivant} & \cellcolor{gray!6}{next}\\

tard/en retard & late\\

\cellcolor{gray!6}{télécharger} & \cellcolor{gray!6}{to download}\\

téléverser & to upload\\

\cellcolor{gray!6}{traiter de} & \cellcolor{gray!6}{to deal with (dealt x 2)}\\

traiter une commande & to run an order\\

\cellcolor{gray!6}{travailleur} & \cellcolor{gray!6}{hard-working}\\

un appareil électronique & a device\\

\cellcolor{gray!6}{un autocollant} & \cellcolor{gray!6}{a sticker}\\

un bâtonnet (yeux) & a rod\\

\cellcolor{gray!6}{un blanchet} & \cellcolor{gray!6}{un blanket}\\

un bobine & a reel\\

\cellcolor{gray!6}{ \vphantom{5}& }\\

\multirow[t]{-2}{*}{\raggedright\arraybackslash un calage} & \multirow[t]{-2}{*}{\raggedright\arraybackslash a make-ready}\\

\cellcolor{gray!6}{ \vphantom{4}& }\\

\multirow[t]{-2}{*}{\raggedright\arraybackslash un client} & \multirow[t]{-2}{*}{\raggedright\arraybackslash a customer}\\

\cellcolor{gray!6}{un dépliant} & \cellcolor{gray!6}{a folded leaflet}\\

un devis & a quote\\

\cellcolor{gray!6}{une caractéristique, une spécification} & \cellcolor{gray!6}{a feature}\\

une cellule & a cell\\

\cellcolor{gray!6}{une couche} & \cellcolor{gray!6}{a layer}\\

un écran (d...) & a display\\

\cellcolor{gray!6}{un écran (m...)} & \cellcolor{gray!6}{a monitor}\\

une encoche & a notch\\

\cellcolor{gray!6}{ \vphantom{3}& }\\

\multirow[t]{-2}{*}{\raggedright\arraybackslash une entreprise} & \multirow[t]{-2}{*}{\raggedright\arraybackslash a company}\\

\cellcolor{gray!6}{une étape} & \cellcolor{gray!6}{a step}\\

une fente & a slit\\

\cellcolor{gray!6}{une forme} & \cellcolor{gray!6}{a shape}\\

une lame & a blade\\

\cellcolor{gray!6}{une livraison} & \cellcolor{gray!6}{a delivery}\\

une machine empileuse & a stacker\\

\cellcolor{gray!6}{un entrepôt} & \cellcolor{gray!6}{a warehouse}\\

une nuance de couleur & a hue\\

\cellcolor{gray!6}{une page} & \cellcolor{gray!6}{a sheet}\\

une plaque & a plate\\

\cellcolor{gray!6}{une poignée} & \cellcolor{gray!6}{a handle}\\

une police (de caractères) & a font\\

\cellcolor{gray!6}{un équilibre} & \cellcolor{gray!6}{a balance}\\

une rainure & a scoring line\\

\cellcolor{gray!6}{une récompense, un prix} & \cellcolor{gray!6}{a prize}\\

une usine & a factory\\

\cellcolor{gray!6}{un logiciel} & \cellcolor{gray!6}{a software}\\

un manchon & a shrinkable sleeve\\

\cellcolor{gray!6}{ \vphantom{2}& }\\

\multirow[t]{-2}{*}{\raggedright\arraybackslash un massicot} & \multirow[t]{-2}{*}{\raggedright\arraybackslash a trimmer}\\

\cellcolor{gray!6}{un métier, commerce} & \cellcolor{gray!6}{a trade}\\

un niveau & a level\\

\cellcolor{gray!6}{un nuancier} & \cellcolor{gray!6}{a fan deck}\\

un plateau & a tray\\

\cellcolor{gray!6}{un pli} & \cellcolor{gray!6}{a fold}\\

un pli-fenêtre & a gate-fold\\

\cellcolor{gray!6}{un pli roulé (UK)} & \cellcolor{gray!6}{a roll fold}\\

un pli roulé (US) & a tri/letter fold\\

\cellcolor{gray!6}{un rabat} & \cellcolor{gray!6}{a flap}\\

un revêtement & a coating\\

\cellcolor{gray!6}{ \vphantom{1}& }\\

\multirow[t]{-2}{*}{\raggedright\arraybackslash un rouleau} & \multirow[t]{-2}{*}{\raggedright\arraybackslash a roller}\\

 & \\

\multirow[t]{-2}{*}{\raggedright\arraybackslash un stage} & \multirow[t]{-2}{*}{\raggedright\arraybackslash an internship}\\

\cellcolor{gray!6}{un trait} & \cellcolor{gray!6}{a stroke}\\

un volume & a form\\

\cellcolor{gray!6}{vernis sélectif} & \cellcolor{gray!6}{spot varnish}\\

vif (couleur) & bright\\
\bottomrule
\end{longtable}

\end{document}
