% Options for packages loaded elsewhere
\PassOptionsToPackage{unicode}{hyperref}
\PassOptionsToPackage{hyphens}{url}
%
\documentclass[
  10pt,
]{article}
\usepackage{lmodern}
\usepackage{amssymb,amsmath}
\usepackage{ifxetex,ifluatex}
\ifnum 0\ifxetex 1\fi\ifluatex 1\fi=0 % if pdftex
  \usepackage[T1]{fontenc}
  \usepackage[utf8]{inputenc}
  \usepackage{textcomp} % provide euro and other symbols
\else % if luatex or xetex
  \usepackage{unicode-math}
  \defaultfontfeatures{Scale=MatchLowercase}
  \defaultfontfeatures[\rmfamily]{Ligatures=TeX,Scale=1}
\fi
% Use upquote if available, for straight quotes in verbatim environments
\IfFileExists{upquote.sty}{\usepackage{upquote}}{}
\IfFileExists{microtype.sty}{% use microtype if available
  \usepackage[]{microtype}
  \UseMicrotypeSet[protrusion]{basicmath} % disable protrusion for tt fonts
}{}
\makeatletter
\@ifundefined{KOMAClassName}{% if non-KOMA class
  \IfFileExists{parskip.sty}{%
    \usepackage{parskip}
  }{% else
    \setlength{\parindent}{0pt}
    \setlength{\parskip}{6pt plus 2pt minus 1pt}}
}{% if KOMA class
  \KOMAoptions{parskip=half}}
\makeatother
\usepackage{xcolor}
\IfFileExists{xurl.sty}{\usepackage{xurl}}{} % add URL line breaks if available
\IfFileExists{bookmark.sty}{\usepackage{bookmark}}{\usepackage{hyperref}}
\hypersetup{
  pdftitle={Le vocabulaire},
  pdfauthor={Adrien Méli},
  hidelinks,
  pdfcreator={LaTeX via pandoc}}
\urlstyle{same} % disable monospaced font for URLs
\usepackage[margin=1in]{geometry}
\usepackage{longtable,booktabs}
% Correct order of tables after \paragraph or \subparagraph
\usepackage{etoolbox}
\makeatletter
\patchcmd\longtable{\par}{\if@noskipsec\mbox{}\fi\par}{}{}
\makeatother
% Allow footnotes in longtable head/foot
\IfFileExists{footnotehyper.sty}{\usepackage{footnotehyper}}{\usepackage{footnote}}
\makesavenoteenv{longtable}
\usepackage{graphicx}
\makeatletter
\def\maxwidth{\ifdim\Gin@nat@width>\linewidth\linewidth\else\Gin@nat@width\fi}
\def\maxheight{\ifdim\Gin@nat@height>\textheight\textheight\else\Gin@nat@height\fi}
\makeatother
% Scale images if necessary, so that they will not overflow the page
% margins by default, and it is still possible to overwrite the defaults
% using explicit options in \includegraphics[width, height, ...]{}
\setkeys{Gin}{width=\maxwidth,height=\maxheight,keepaspectratio}
% Set default figure placement to htbp
\makeatletter
\def\fps@figure{htbp}
\makeatother
\setlength{\emergencystretch}{3em} % prevent overfull lines
\providecommand{\tightlist}{%
  \setlength{\itemsep}{0pt}\setlength{\parskip}{0pt}}
\setcounter{secnumdepth}{5}
\usepackage[T1]{fontenc}
\usepackage[utf8]{inputenc}
%\usepackage{fontspec}
\usepackage{fontawesome}
\usepackage{fancyhdr}
\usepackage{tipa}
\usepackage{hyperref}
%\usepackage{booktabs}
\usepackage{titlesec}
\usepackage{multicol}
\usepackage{float}
\usepackage{colortbl}
%\usepackage{tufte-book}
\usepackage{xcolor}
%\usepackage{fourier}
%\usepackage{montserrat}
%\usepackage[french]{babel}
\usepackage[modulo]{lineno}
\usepackage{tikz}
\usepackage{graphicx}
\usepackage[light, sfdefault]{roboto}

%\renewcommand*\familydefault{\sfdefault} 
\renewcommand{\rmdefault}{ptm}
%----------------------------------------------------------------------------------------
%	DEFINE COLOURS
%----------------------------------------------------------------------------------------
%\definecolor{darkPrimaryColor}{HTML}{303F9F}
\definecolor{darkPrimaryColor}{HTML}{2c5272}
\definecolor{primaryColor}{HTML}{3F51B5}
\definecolor{lightPrimaryColor}{HTML}{C5CAE9}
\definecolor{textPrimaryColor}{HTML}{FFFFFF}
\definecolor{accentColor}{HTML}{FF5252}
\definecolor{primaryTextColor}{HTML}{212121}
\definecolor{secondaryTextColor}{HTML}{757575}
\definecolor{dividerColor}{HTML}{BDBDBD}

% -----------------------------------------------------------------
% Hyper Setup
% -----------------------------------------------------------------
\hypersetup{
    %bookmarks=true,         % show bookmarks bar?
    unicode=false,          % non-Latin characters in Acrobat�s bookmarks
    pdftoolbar=true,        % show Acrobat�s toolbar?
    pdfmenubar=true,        % show Acrobat�s menu?
    pdffitwindow=false,     % window fit to page when opened
    pdfstartview={FitH},    % fits the width of the page to the window
    %pdftitle={},    % title
    pdfauthor={Adrien Meli},     % author 
    pdfsubject={Phonological rules},   % subject of the document
    pdfcreator={Creator},   % creator of the document
    pdfproducer={Producer}, % producer of the document
    pdfkeywords={Second Language Acquisition, French, English, phonology}, % list of keywords
    pdfnewwindow=true,     % links in new window
    colorlinks=true,       % false: boxed links; true: colored links
    linkcolor=black,          % color of internal links
    citecolor=black,        % color of links to bibliography
    filecolor=magenta,      % color of file links
    urlcolor=blue,           % color of external links
    bookmarksopen=false,
    anchorcolor=black,
    bookmarksnumbered=true,
    pdfpagemode=UseOutlines,    %None/UseOutlines/UseThumbs/FullScreen
    linktocpage=true
}

% ********************Captions and Hyperreferencing / URL **********************

% Captions: This makes captions of figures use a boldfaced small font.

\usepackage[margin=10pt,font=small,labelfont=bf,labelsep=endash]{caption}

% -----------------------------------------------------------------
% TABLE OF CONTENTS
% -----------------------------------------------------------------
\usepackage[dotinlabels]{titletoc}
\titlecontents{section}[0em] % entries are pushed to the rtight
  {} % code to change the appearance
  {} % section number: increase distance to push to the left
  {\hspace*{3.3em}}
  {\titlerule*[1.9mm]{.}\contentspage}
% remove subsections from TOC
\setcounter{tocdepth}{1}

% multi-line curly brackets


%----------------------------------------------------------------------------------------
%	MODIFY SECTION STYLES
%----------------------------------------------------------------------------------------

\usepackage{titlesec} % Required for modifying sections

%------------------------------------------------
% Section

\titleformat
   {\section} % Section type being modified
   [block] % Shape type, can be: hang, block, display, runin, leftmargin, rightmargin, drop, wrap, frame
   {\bfseries\large} % Format of the whole section
   {\thesection.} % Format of the section label
   {6pt} % Space between the title and label
   %{\titlerule\newline{\thetitle. }} % Code before the label
   {\begin{nolinenumbers}} % Code before the label
   [\end{nolinenumbers}]% after code

\titlespacing{\section}
   {0pt} % left
   {\baselineskip} % before title
   {\baselineskip} % Spacing around section titles, the order is: left, before and after

%------------------------------------------------
% Subsection

\titleformat
   {\subsection}
   [block] % Shape type, can be: hang, block, display, runin, leftmargin, rightmargin, drop, wrap, frame
   {\bfseries }% format
   %{\thesubsection.}% label
   {\thesubsection.}% label
   {6pt}% sep btw label and title
   {\begin{nolinenumbers}} % Code before the label
   [\end{nolinenumbers}]% after code
 
\titlespacing{\subsection}
   {6pt}% left
   {\baselineskip} % before title
   {\baselineskip} % Spacing around section titles, the order is: left, before and after

%------------------------------------------------
% Subsubsection

\titleformat
   {\subsubsection}
   [block] % Shape type, can be: hang, block, display, runin, leftmargin, rightmargin, drop, wrap, frame
   {\itshape}% format
   {}% label
   {6pt}% sep btw label and title
   {\begin{nolinenumbers}} % Code before the label
   [\end{nolinenumbers}]% after code
 
\titlespacing{\subsubsection}
   {12pt}% left
   {\baselineskip} % before title
   {\baselineskip}% after sep

%\renewcommand\thesubsection{(\alph{subsection})}

% from bookdown to create column environments
\newenvironment{columns}[1][]{}{}

\newenvironment{column}[1]{\begin{minipage}{#1}\ignorespaces}{%
\end{minipage}
\ifhmode\unskip\fi
\aftergroup\useignorespacesandallpars}

\def\useignorespacesandallpars#1\ignorespaces\fi{%
#1\fi\ignorespacesandallpars}

\makeatletter
\def\ignorespacesandallpars{%
  \@ifnextchar\par
    {\expandafter\ignorespacesandallpars\@gobble}%
    {}%
}
\makeatother


% -----------------------------------------------------------------
% TITLE
% -----------------------------------------------------------------

\title{Mon super template}
\author{Adrien M\'{e}li\\{\tt adrienmeli@gmail.com}}
\date{\today}


% -----------------------------------------------------------------
% DEFINING HEADERS
% -----------------------------------------------------------------

\lhead{}
\chead{}
\rhead{\tiny{\today}}
\lfoot{}
\cfoot{\thepage}
\rfoot{\tiny Adrien M\'{e}li \textcopyright \the\year}
\renewcommand{\headrulewidth}{0.1mm}
\renewcommand{\footrulewidth}{0.1mm}

% --------------------------------------
% TO INTEGRATE THE TITLE.TEX
% -------------------------------------
\setlength{\headheight}{14pt}
\newcommand{\HRule}{\rule{\linewidth}{0.5mm}}
\pagestyle{fancy}
\ifluatex
  \usepackage{selnolig}  % disable illegal ligatures
\fi

\title{Le vocabulaire}
\author{Adrien Méli}
\date{November 04, 2020}

\begin{document}
\maketitle

\pagestyle{fancy}
%\begin{linenumbers}[1]
%  \modulolinenumbers[5]

\hypertarget{introduction}{%
\section{Introduction}\label{introduction}}

Comment constituer une liste de vocabulaire anglais ?

Le vocabulaire s'apprend plus efficacement si la liste de mots reflète certaines caractéristiques grammaticales.

Voici pour les principales catégories grammaticales quelques conseils sur les informations à inclure dans votre liste de vocabulaire.
Ces conseils sont assortis de quelques rappels et astuces.

\hypertarget{les-noms-anglais}{%
\section{Les noms anglais}\label{les-noms-anglais}}

En anglais, certains noms peuvent se mettre au pluriel, d'autres ne le peuvent pas.

Les premiers sont dits ``\textbf{comptables},'' les deuxièmes, ``\textbf{incomptables}.''

\hypertarget{les-noms-comptables}{%
\subsection{Les noms comptables}\label{les-noms-comptables}}

Ces noms prennent obligatoirement un article au singulier.
Ils se mettent généralement au pluriel avec le suffixe \textless-(e)s\textgreater.

\begin{quote}
\textbf{Conseil :} Dans votre liste de vocabulaire, faites précéder les noms comptables de l'article indéfini ``\emph{a}.''

\emph{Exemple : ``a tool'' \(\rightarrow\)} ``un outil.''
\end{quote}

\hypertarget{les-noms-incomptables}{%
\subsection{Les noms incomptables}\label{les-noms-incomptables}}

Ces noms ne peuvent pas se mettre au pluriel.

\begin{quote}
\textbf{Conseil :} Dans votre liste de vocabulaire, indiquez les noms incomptables par ``\emph{(U)},'' pour ``\emph{uncountable}.''

\emph{Exemple : ``evidence (U)'' \(\rightarrow\)} ``des preuves.''
\end{quote}

\hypertarget{bon-uxe0-savoir}{%
\subsection{Bon à savoir}\label{bon-uxe0-savoir}}

\hypertarget{mots-courants-incomptables}{%
\subsubsection{Mots courants incomptables}\label{mots-courants-incomptables}}

Attention aux noms suivants, qui ne se mettent pas au pluriel en anglais :

\begin{itemize}
\tightlist
\item
  \emph{information (U)}
\item
  \emph{furniture (U)} (``meuble'')
\item
  \emph{advice (U)} (``conseil'')
\end{itemize}

Pour dire ``un meuble'' ou ``un conseil,'' on dira ``\emph{a piece of furniture}'' ou ``\emph{a piece of advice}.''

\hypertarget{singuliers-avec--s}{%
\subsubsection{Singuliers avec \textless-s\textgreater{}}\label{singuliers-avec--s}}

Les mots suivants ont un \textless-s\textgreater{} au singulier :

\begin{itemize}
\tightlist
\item
  \emph{a series} (``une série'')
\item
  \emph{a means} (``un moyen'')
\item
  \emph{a species} (``une espèce'')
\end{itemize}

Attention aussi à \color[HTML]{f44336}\emph{the news} \color{black} (``les nouvelles''), qui est \textbf{toujours singulier}.

\hypertarget{pluriels-sans--s}{%
\subsubsection{Pluriels sans \textless-s\textgreater{}}\label{pluriels-sans--s}}

Inversement, il existe des noms qui peuvent ou doivent se conjuguer au pluriel :

\begin{itemize}
\tightlist
\item
  \color[HTML]{f44336}\emph{the police} \color{black} est \textbf{toujours pluriel}.
\item
  \emph{staff}, \emph{team} ou \emph{family} peuvent se conjuguer au pluriel.
\end{itemize}

\hypertarget{les-adjectifs}{%
\section{Les adjectifs}\label{les-adjectifs}}

Les adjectifs anglais sont \textbf{invariables}, et se placent \textbf{avant} le nom.

\begin{itemize}
\tightlist
\item
  ``Des maisons bleues'' \(\rightarrow\) \emph{blue houses}.
\end{itemize}

\begin{quote}
\textbf{Conseil :} Dans votre liste de vocabulaire, indiquez bien la préposition avec laquelle l'adjectif se construit.

\emph{Exemple : ``similar \textbf{to}'' \(\rightarrow\)} ``semblable à.''
\end{quote}

Voici quelques exemples d'adjectifs fréquents se construisant avec des prépositions différentes du français :

\begin{itemize}
\tightlist
\item
  \emph{responsible \textbf{for}} \(\rightarrow\) ``responsable de''
\item
  \emph{dependent \textbf{on}} \(\rightarrow\) ``dépendant de''
\item
  \emph{addicted \textbf{to}} \(\rightarrow\) ``accroc à''
\item
  \emph{interested \textbf{in}} \(\rightarrow\) ``intéressé par''
\item
  \emph{good \textbf{at}} \(\rightarrow\) ``bon en''
\item
  \emph{different \textbf{from}} \(\rightarrow\) ``différent de''
\item
  \emph{worried \textbf{about}} \(\rightarrow\) ``inquiet de''
\item
  \emph{surprised \textbf{at}} \(\rightarrow\) ``surpris par''
\end{itemize}

\hypertarget{les-verbes}{%
\section{Les verbes}\label{les-verbes}}

\begin{quote}
\textbf{Conseil :} Dans votre liste de vocabulaire, faites précéder les verbes de ``\emph{to}'' afin de les distinguer des autres catégories grammaticales.

\emph{Exemple : ``to row'' \(\rightarrow\)} ``ramer''
\end{quote}

Comme les adjectifs, les verbes s'apprennent avec leur \textbf{construction}.

\hypertarget{les-verbes-transitifs}{%
\subsection{Les verbes transitifs}\label{les-verbes-transitifs}}

On distingue les \textbf{verbes transitifs directs}, qui admettent un complément d'objet direct, des \textbf{verbes transitifs indirects}, dont le complément
est séparé du verbe par une préposition.

Considérez les exemples suivants :

\begin{itemize}
\tightlist
\item
  \emph{to listen \textbf{to} stg} \(\rightarrow\) ``écouter qqch''
\item
  \emph{to look \textbf{for} stg} \(\rightarrow\) ``chercher qqch''
\item
  \emph{to look \textbf{at} stg} \(\rightarrow\) ``regarder qqch''
\end{itemize}

Dans ces exemples, le verbe anglais est transitif indirect, tandis que le verbe français est transitif direct.

À l'inverse :

\begin{itemize}
\tightlist
\item
  \emph{to obey sb} \(\rightarrow\) ``obéir \textbf{à} qn''
\end{itemize}

\begin{quote}
\textbf{Conseil :} Dans votre liste de vocabulaire, indiquez la construction du verbe avec ``\emph{stg}'' (``\emph{something}'') ou ``\emph{sb}'' (``\emph{somebody}'').

\emph{Exemple : ``to blame sb for stg'' \(\rightarrow\)} ``reprocher qqch à qn''
\end{quote}

\hypertarget{les-verbes-intransitifs}{%
\subsection{Les verbes intransitifs}\label{les-verbes-intransitifs}}

Ces verbes se contruisent sans complément, et par conséquent ne peuvent se mettre au passif.

\emph{Exemples : ``to rain,'' ``to lie,'' ``to rise''}.

\hypertarget{bon-uxe0-savoir-1}{%
\subsection{Bon à savoir}\label{bon-uxe0-savoir-1}}

\hypertarget{une-confusion-fruxe9quente}{%
\subsubsection{Une confusion fréquente}\label{une-confusion-fruxe9quente}}

Ne confondez plus :

\begin{itemize}
\tightlist
\item
  ``\emph{rise (rose risen)}'' et ``\emph{raise}'' ;
\item
  ``\emph{lie (lay lain)}'' et ``\emph{lay}.''
\end{itemize}

\begin{longtable}[]{@{}cc@{}}
\toprule
\begin{minipage}[b]{0.48\columnwidth}\centering
\textbf{Intransitif}\strut
\end{minipage} & \begin{minipage}[b]{0.46\columnwidth}\centering
\textbf{Transitif}\strut
\end{minipage}\tabularnewline
\midrule
\endhead
\begin{minipage}[t]{0.48\columnwidth}\centering
\emph{to rise (rose risen)} \(\rightarrow\) ``élever''\strut
\end{minipage} & \begin{minipage}[t]{0.46\columnwidth}\centering
\emph{to raise stg} \(\rightarrow\) ``lever qqch''\strut
\end{minipage}\tabularnewline
\begin{minipage}[t]{0.48\columnwidth}\centering
\emph{to lie (lay lain)} \(\rightarrow\) ``se situer''\strut
\end{minipage} & \begin{minipage}[t]{0.46\columnwidth}\centering
\emph{to lay stg} \(\rightarrow\) ``poser qqch''\strut
\end{minipage}\tabularnewline
\bottomrule
\end{longtable}

Notez aussi que les deux intransitifs, ``\emph{rise}'' et ``\emph{lie},'' se prononcent avec /\textipa{aI}/ comme
dans ``\emph{high}.''

Les deux transitifs se prononcent eux avec /\textipa{eI}/, comme dans ``\emph{say}.''

\hypertarget{les-adverbes}{%
\section{Les adverbes}\label{les-adverbes}}

Les adverbes fonctionnent comme en français : ils qualifient généralement un verbe et son complément de la même manière qu'un adjectif qualifie un nom.

Attention à l'emplacement des adverbes dits ``de fréquence,'' qui s'insèrent souvent entre le sujet et le verbe conjugué :

\begin{itemize}
\tightlist
\item
  \emph{He \textbf{often} speaks to himself} \(\rightarrow\) Il se parle souvent à lui-même.
\end{itemize}

\hypertarget{les-pruxe9positions}{%
\section{Les prépositions}\label{les-pruxe9positions}}

(W.I.P.)

Les prépositions sont des mots suivis d'un Groupe Nominal, ou d'une forme verbale en \textless-ING\textgreater.

\hypertarget{vocabulaire}{%
\section{Vocabulaire}\label{vocabulaire}}

\hypertarget{dn1}{%
\subsection{DN1}\label{dn1}}

\begin{longtable}{ll}
\toprule
Français & English\\
\midrule
\cellcolor{gray!6}{15 sur 20} & \cellcolor{gray!6}{15 out of 20}\\

abandonner & to give up\\

\cellcolor{gray!6}{adhérer à qqch} & \cellcolor{gray!6}{to subscribe to stg}\\

alors que & whereas\\

\cellcolor{gray!6}{améliorer, faire des progrès} & \cellcolor{gray!6}{to improve}\\

appartenir à & to belong to\\

\cellcolor{gray!6}{approvisionner, fournir} & \cellcolor{gray!6}{to supply}\\

attendre de qn qu'il fasse qqch & to expect sb to do stg\\

\cellcolor{gray!6}{bien que} & \cellcolor{gray!6}{although}\\

capacité d'attention & attention span\\

\cellcolor{gray!6}{cautionner} & \cellcolor{gray!6}{to endorse}\\

concret & hands-on\\

\cellcolor{gray!6}{de la boue} & \cellcolor{gray!6}{mud}\\

des aiguilles à tricoter & knitting needles\\

\cellcolor{gray!6}{du mobilier, des meubles} & \cellcolor{gray!6}{furniture}\\

écouter qqch & to listen to stg\\

\cellcolor{gray!6}{expliciter, détailler} & \cellcolor{gray!6}{to spell out}\\

fierté, orgueil & pride\\

\cellcolor{gray!6}{fonder} & \cellcolor{gray!6}{to found}\\

froncer les sourcils, désapprouver qqch & to frown (on stg)\\

\cellcolor{gray!6}{grandir} & \cellcolor{gray!6}{to grow up}\\

interdire (b...) & to ban\\

\cellcolor{gray!6}{interdire (f...)} & \cellcolor{gray!6}{to forbid}\\

interdire (p...) & to prohibit\\

\cellcolor{gray!6}{le siège d'une entreprise} & \cellcolor{gray!6}{the headquarters}\\

mais, pourtant & yet\\

\cellcolor{gray!6}{même si (concession)} & \cellcolor{gray!6}{even though}\\

mettre en oeuvre & to implement\\

\cellcolor{gray!6}{obligatoire} & \cellcolor{gray!6}{compulsory}\\

pendant que & while\\

\cellcolor{gray!6}{pouvoir s'acheter qqch} & \cellcolor{gray!6}{to afford something}\\

réclamer, exiger & to call for\\

\cellcolor{gray!6}{rendre un hommage} & \cellcolor{gray!6}{to pay a tribute}\\

rester, demeurer & to remain\\

\cellcolor{gray!6}{s'avérer} & \cellcolor{gray!6}{to turn out}\\

se concentrer sur qqch & to focus on stg\\

\cellcolor{gray!6}{s'empresser de} & \cellcolor{gray!6}{to rush to do stg}\\

stable & steady\\

\cellcolor{gray!6}{une politique, une mesure} & \cellcolor{gray!6}{a policy}\\

une usine & a factory\\

\cellcolor{gray!6}{un intrus} & \cellcolor{gray!6}{an odd-one-out}\\

un noyau & a core\\

\cellcolor{gray!6}{un ordinateur de bureau} & \cellcolor{gray!6}{a desktop}\\

un ordinateur portable & a laptop\\

\cellcolor{gray!6}{un outil} & \cellcolor{gray!6}{a tool}\\

un pilier & a pillar\\

\cellcolor{gray!6}{un portail} & \cellcolor{gray!6}{a gate}\\

un résumé & a summary\\

\cellcolor{gray!6}{un sondage} & \cellcolor{gray!6}{a poll}\\

un vendeur au détail & a retailer\\

\cellcolor{gray!6}{un vif désir} & \cellcolor{gray!6}{a compulsion}\\

conseiller & to advise sb to do stg\\

\cellcolor{gray!6}{les sous-titres} & \cellcolor{gray!6}{the subtitles}\\

avoir hâte de & to look forward to + -ING\\

\cellcolor{gray!6}{être composé de} & \cellcolor{gray!6}{to be made up of}\\

être agenouillé & to kneel (knelt x 2)\\

\cellcolor{gray!6}{un pantalon} & \cellcolor{gray!6}{a pair of trousers}\\

un vol & a theft\\

\cellcolor{gray!6}{le personnel} & \cellcolor{gray!6}{the staff}\\

un atelier & a workshop\\

\cellcolor{gray!6}{un conservateur de musée} & \cellcolor{gray!6}{a curator}\\

un cadre & a frame\\

\cellcolor{gray!6}{un indice} & \cellcolor{gray!6}{a clue}\\

une exposition & an exhibition\\

\cellcolor{gray!6}{exposer} & \cellcolor{gray!6}{to display}\\

remarquer & to notice\\

\cellcolor{gray!6}{perdu} & \cellcolor{gray!6}{unaccounted for}\\

faire attention à qqch & to pay attention to\\

\cellcolor{gray!6}{rendre visite à} & \cellcolor{gray!6}{to pay a visit to}\\

sombre & dark\\

\cellcolor{gray!6}{vif, éclatant} & \cellcolor{gray!6}{bright}\\

un défaut & a drawback\\

\cellcolor{gray!6}{au lieu de} & \cellcolor{gray!6}{instead of}\\
\bottomrule
\end{longtable}

\hypertarget{dn2}{%
\subsection{DN2}\label{dn2}}

\begin{longtable}{ll}
\toprule
Français & English\\
\midrule
\cellcolor{gray!6}{flou} & \cellcolor{gray!6}{blurry}\\

rugueux & rough\\

\cellcolor{gray!6}{doux (au toucher)} & \cellcolor{gray!6}{smooth}\\

brouillard & fog\\

\cellcolor{gray!6}{brume} & \cellcolor{gray!6}{mist}\\

la partie inférieure & the bottom part\\

\cellcolor{gray!6}{la partie supérieure} & \cellcolor{gray!6}{the upper part}\\

aborder (un sujet) & to tackle\\

\cellcolor{gray!6}{à haute criminalité} & \cellcolor{gray!6}{crime-ridden}\\

a ridge & une crête\\

\cellcolor{gray!6}{atroce, ou criard} & \cellcolor{gray!6}{lurid}\\

attendre de qn que... & to expect sb to\\

\cellcolor{gray!6}{au premier plan} & \cellcolor{gray!6}{in the foreground}\\

biaisé, partial & biassed\\

\cellcolor{gray!6}{ce qui est produit, ce qui sort} & \cellcolor{gray!6}{the output}\\

chaos & mayhem\\

\cellcolor{gray!6}{comestible} & \cellcolor{gray!6}{edible}\\

destinataire, récipiendaire & a recipient\\

\cellcolor{gray!6}{diffuser} & \cellcolor{gray!6}{to broadcast}\\

disposer, agencer & to lay out\\

\cellcolor{gray!6}{durer} & \cellcolor{gray!6}{to last}\\

échapper à & to elude\\

\cellcolor{gray!6}{esquisser} & \cellcolor{gray!6}{to sketch}\\

être condamné & to be doomed\\

\cellcolor{gray!6}{être susceptible de} & \cellcolor{gray!6}{to be likely to}\\

évident & obvious\\

\cellcolor{gray!6}{exact, précis} & \cellcolor{gray!6}{accurate}\\

faire 1m. de hauteur & to be 1m. high\\

\cellcolor{gray!6}{faire 1m. de large} & \cellcolor{gray!6}{to be 1m. wide}\\

faire 1m. de profondeur & to be 1m. deep\\

\cellcolor{gray!6}{fournir (p...)} & \cellcolor{gray!6}{to provide sb with}\\

fournir (s...) & to supply sb with\\

\cellcolor{gray!6}{il semble que} & \cellcolor{gray!6}{it looks as if}\\

une tache, une zone & a patch\\

\cellcolor{gray!6}{incliner} & \cellcolor{gray!6}{to tilt}\\

intenter un procès & to sue\\

\cellcolor{gray!6}{interdire} & \cellcolor{gray!6}{to ban}\\

la majorité écrasante & the overwhelming majority\\

\cellcolor{gray!6}{la poitrine} & \cellcolor{gray!6}{the chest}\\

la respiration, l'haleine & the breath\\

\cellcolor{gray!6}{le confinement} & \cellcolor{gray!6}{the lockdown}\\

le coucher du soleil & the sunset\\

\cellcolor{gray!6}{le sein maternel} & \cellcolor{gray!6}{the bosom}\\

menacer & to threaten\\

\cellcolor{gray!6}{pendant que} & \cellcolor{gray!6}{while}\\

posséder qqch & to own stg\\

\cellcolor{gray!6}{postuler à qqch} & \cellcolor{gray!6}{to apply for stg}\\

promouvoir (a...) & to advertise\\

\cellcolor{gray!6}{promouvoir (p...)} & \cellcolor{gray!6}{to promote}\\

provenir de & to stem from\\

\cellcolor{gray!6}{résoudre} & \cellcolor{gray!6}{to work out}\\

s'accroupir & to croush\\

\cellcolor{gray!6}{selon, d'après} & \cellcolor{gray!6}{according to}\\

se noyer & to drown\\

\cellcolor{gray!6}{(se) terminer} & \cellcolor{gray!6}{to be over}\\

sonder & to probe\\

\cellcolor{gray!6}{un but, un objectif (a...)} & \cellcolor{gray!6}{an aim}\\

un but, un objectif (p...) & a purpose\\

\cellcolor{gray!6}{un contour} & \cellcolor{gray!6}{an outline}\\

un écart (d...) & a discrepancy\\

\cellcolor{gray!6}{un écart (g...)} & \cellcolor{gray!6}{a gap}\\

une épave de bateau & a shipwreck\\

\cellcolor{gray!6}{une légende d'image} & \cellcolor{gray!6}{a caption}\\

une mission, une tâche & an assignment\\

\cellcolor{gray!6}{une pièce de 10 ¢} & \cellcolor{gray!6}{a dime}\\

une pièce de 25 ¢ & a quarter\\

\cellcolor{gray!6}{une pièce de 5 ¢} & \cellcolor{gray!6}{a nickel}\\

une tâche de couleur & a patch\\

\cellcolor{gray!6}{une tempête} & \cellcolor{gray!6}{a storm}\\

une toile & a canvas\\

\cellcolor{gray!6}{une vague} & \cellcolor{gray!6}{a wave}\\

une vue d'ensemble & an overview\\

\cellcolor{gray!6}{un hommage} & \cellcolor{gray!6}{a tribute}\\

un indice & a clue\\

\cellcolor{gray!6}{un ordinateur de bureau} & \cellcolor{gray!6}{a desktop}\\

un ordinateur portable & a laptop\\

\cellcolor{gray!6}{un rédacteur en chef} & \cellcolor{gray!6}{an editor}\\

un sondage & a poll\\

\cellcolor{gray!6}{un stage} & \cellcolor{gray!6}{an internship}\\

un traité (livre) & a treatise\\

\cellcolor{gray!6}{vendre la mèche} & \cellcolor{gray!6}{to spill the beans}\\

élargir & to widen\\

\cellcolor{gray!6}{approfondir} & \cellcolor{gray!6}{to deepen}\\

rehausser & to heighten\\

\cellcolor{gray!6}{allonger} & \cellcolor{gray!6}{to lengthen}\\

largeur & width\\

\cellcolor{gray!6}{longueur} & \cellcolor{gray!6}{length}\\

profondeur & depth\\

\cellcolor{gray!6}{hauteur} & \cellcolor{gray!6}{height}\\

riche, abondant, aisé & wealthy\\

\cellcolor{gray!6}{concret} & \cellcolor{gray!6}{hands-on}\\

un devoir & a duty\\

\cellcolor{gray!6}{la sagesse} & \cellcolor{gray!6}{wisdom}\\

être excité à l'idée de & to be hung up on...\\

\cellcolor{gray!6}{l'état-providence} & \cellcolor{gray!6}{the welfare state}\\

le noyau & the core\\

\cellcolor{gray!6}{une revendication} & \cellcolor{gray!6}{a claim}\\

recouper, inclure & to cut across\\

\cellcolor{gray!6}{un festin} & \cellcolor{gray!6}{a feast}\\

mêler à & to embroil\\

\cellcolor{gray!6}{supporter, accepter} & \cellcolor{gray!6}{to bear}\\

cinglant & searing\\

\cellcolor{gray!6}{une hypothèse} & \cellcolor{gray!6}{an assumption}\\

sans fard, simple et clair & bald\\

\cellcolor{gray!6}{une tentative} & \cellcolor{gray!6}{an endeavour}\\

lutter & to struggle\\

\cellcolor{gray!6}{vaincre} & \cellcolor{gray!6}{to overcome}\\

une limite, une frontière & a border\\

\cellcolor{gray!6}{superficiel, peu profond} & \cellcolor{gray!6}{shallow}\\

une énigme & a riddle\\
\bottomrule
\end{longtable}

\hypertarget{dn3}{%
\subsection{DN3}\label{dn3}}

\begin{longtable}{ll}
\toprule
Français & English\\
\midrule
\cellcolor{gray!6}{aborder un problème} & \cellcolor{gray!6}{to address an issue}\\

augmenter & to increase\\

\cellcolor{gray!6}{bien que} & \cellcolor{gray!6}{although}\\

dans quelle mesure & to what extent\\

\cellcolor{gray!6}{de plus} & \cellcolor{gray!6}{furthermore}\\

dès le départ & right off the bat\\

\cellcolor{gray!6}{en dépit de} & \cellcolor{gray!6}{despite}\\

exécution, mise en œuvre & implementation\\

\cellcolor{gray!6}{les résultats} & \cellcolor{gray!6}{the findings}\\

mettre à nu & to lay bare\\

\cellcolor{gray!6}{pertinent} & \cellcolor{gray!6}{relevant}\\

puisque & since\\

\cellcolor{gray!6}{se concentrer sur} & \cellcolor{gray!6}{to focus on}\\

un but (p...) & a purpose\\

\cellcolor{gray!6}{un échantillon} & \cellcolor{gray!6}{a sample}\\

une tendance & a trend\\

\cellcolor{gray!6}{un moyen de} & \cellcolor{gray!6}{a means to}\\

un résultat (o...) & an outcome\\

\cellcolor{gray!6}{un résumé} & \cellcolor{gray!6}{a summary}\\

viser à, avoir pour but de & to aim to\\

\cellcolor{gray!6}{gâcher} & \cellcolor{gray!6}{to waste}\\

un conseil (t...) & a tip\\

\cellcolor{gray!6}{un conseil (a...)} & \cellcolor{gray!6}{advice}\\

une poubelle (UK) & a rubbish bin\\

\cellcolor{gray!6}{une poubelle (US)} & \cellcolor{gray!6}{a trash can}\\

évaluer & to assess\\

\cellcolor{gray!6}{réparer} & \cellcolor{gray!6}{to fix}\\

se plaindre, geindre & to whine\\

\cellcolor{gray!6}{stupéfier, "scotcher"} & \cellcolor{gray!6}{to blow}\\

résoudre & to solve\\

\cellcolor{gray!6}{fournir qqch à qn (p...)} & \cellcolor{gray!6}{to provide sb with}\\

fournir qqch à qn (s...) & to supply sb with\\

\cellcolor{gray!6}{actuel} & \cellcolor{gray!6}{current}\\

une étude, un sondage & a survey\\
\bottomrule
\end{longtable}

\hypertarget{erpc-1uxe8re-annuxe9e}{%
\subsection{ERPC 1ère année}\label{erpc-1uxe8re-annuxe9e}}

\begin{longtable}{ll}
\toprule
Français & English\\
\midrule
\cellcolor{gray!6}{alimenter} & \cellcolor{gray!6}{to feed (fed, fed)}\\

à travers quelque chose & through\\

\cellcolor{gray!6}{autocollants} & \cellcolor{gray!6}{stickers}\\

brochures & booklets\\

\cellcolor{gray!6}{bye} & \cellcolor{gray!6}{au revoir}\\

cartes de visite & business cards\\

\cellcolor{gray!6}{charger} & \cellcolor{gray!6}{to load}\\

commencer & to begin (began, begun)\\

\cellcolor{gray!6}{de la poudre} & \cellcolor{gray!6}{powder}\\

de l'encre & ink\\

\cellcolor{gray!6}{de l'huile} & \cellcolor{gray!6}{oil}\\

dépliants & brochures\\

\cellcolor{gray!6}{des agrafes} & \cellcolor{gray!6}{staples}\\

deux fois & twice\\

\cellcolor{gray!6}{dos carré-collé} & \cellcolor{gray!6}{perfect binding}\\

équipement, installation & facility\\

\cellcolor{gray!6}{étape} & \cellcolor{gray!6}{a stage}\\

expédier & to ship out\\

\cellcolor{gray!6}{fonctionner} & \cellcolor{gray!6}{to work}\\

glisser & to glide\\

\cellcolor{gray!6}{gravé au laser} & \cellcolor{gray!6}{laser-etched}\\

hello & bonjour\\

\cellcolor{gray!6}{humecter, humidifier} & \cellcolor{gray!6}{to dampen}\\

imprimer & to print\\

\cellcolor{gray!6}{item fonctionner} & \cellcolor{gray!6}{to work}\\

le dos (d'un livre) & the spine\\

\cellcolor{gray!6}{le grammage} & \cellcolor{gray!6}{paper weight}\\

le recto & the front\\

\cellcolor{gray!6}{le verso} & \cellcolor{gray!6}{the back}\\

livrer & to deliver\\

\cellcolor{gray!6}{mieux convenir à} & \cellcolor{gray!6}{to be best suited for}\\

mince, fin & thin\\

\cellcolor{gray!6}{pailleté, miroitant} & \cellcolor{gray!6}{shimmery}\\

piqûre à cheval & saddle-stitched\\

\cellcolor{gray!6}{précis, aiguisé} & \cellcolor{gray!6}{sharp}\\

recto-verso & both sides\\

\cellcolor{gray!6}{relier (un livre)} & \cellcolor{gray!6}{to bind (bound, bound)}\\

reliure à spirale & coil binding\\

\cellcolor{gray!6}{résulter dans, aboutir à} & \cellcolor{gray!6}{to result in}\\

sans & without\\

\cellcolor{gray!6}{stokage} & \cellcolor{gray!6}{storage}\\

taille & size\\

\cellcolor{gray!6}{tomber, chuter} & \cellcolor{gray!6}{to fall (fell, fallen)}\\

trade & commerce\\

\cellcolor{gray!6}{une agence de communication} & \cellcolor{gray!6}{an ad(vertising) agency}\\

une configuration & a setup\\

\cellcolor{gray!6}{une couche} & \cellcolor{gray!6}{a \vphantom{1} layer}\\


une couverture, un blanchet & a blanket\\

\cellcolor{gray!6}{une encoche} & \cellcolor{gray!6}{a notch}\\

une enveloppe & a wrap\\

\cellcolor{gray!6}{une fente} & \cellcolor{gray!6}{a slit}\\

une finition brillante & a glossy finish\\

\cellcolor{gray!6}{une finition mate} & \cellcolor{gray!6}{a matte finish}\\

une fois que & once\\

\cellcolor{gray!6}{une plaque} & \cellcolor{gray!6}{a plate}\\

une sous-couche & an under-coat\\

\cellcolor{gray!6}{une surface, une tache} & \cellcolor{gray!6}{a spot}\\

un exemplaire & a copy\\

\cellcolor{gray!6}{un fichier numérique} & \cellcolor{gray!6}{a digital file}\\

un massicot & a trimmer\\

\cellcolor{gray!6}{un pli} & \cellcolor{gray!6}{a fold}\\

un rouleau & a roller\\

\cellcolor{gray!6}{un sondage} & \cellcolor{gray!6}{a poll}\\

un stage & an internship\\

\cellcolor{gray!6}{piqûre à cheval} & \cellcolor{gray!6}{saddle-stitching}\\

reliure spirales & coil binding\\

\cellcolor{gray!6}{expédier qqch} & \cellcolor{gray!6}{to ship out stg}\\

une nuance de couleur & a hue\\

\cellcolor{gray!6}{délavé} & \cellcolor{gray!6}{washed out}\\

mat & matte\\

\cellcolor{gray!6}{brillant} & \cellcolor{gray!6}{glossy}\\

une couche & a layer\\
\cellcolor{gray!6}{mélanger} & \cellcolor{gray!6}{to mix}\\
\bottomrule
\end{longtable}

\hypertarget{erpc-2uxe8me-annuxe9e}{%
\subsection{ERPC 2ème année}\label{erpc-2uxe8me-annuxe9e}}

\begin{longtable}{ll}
\toprule
Français & \vphantom{1} English\\

\midrule
\cellcolor{gray!6}{ouvert d'esprit} & \cellcolor{gray!6}{open-minded}\\

Français & English\\
\cellcolor{gray!6}{PAO} & \cellcolor{gray!6}{Desktop Publishing (DTP)}\\

aimer faire qqch & to like doing stg\\

\cellcolor{gray!6}{ajouter} & \cellcolor{gray!6}{to add}\\

assister à une réunion & to attend a meeting\\

\cellcolor{gray!6}{attentionné} & \cellcolor{gray!6}{caring}\\

avoir peur de qqch & to be afraid of stg\\

\cellcolor{gray!6}{bavard} & \cellcolor{gray!6}{chatty}\\

bien aimer faire qqch & to enjoy doing stg\\

\cellcolor{gray!6}{bien conçu} & \cellcolor{gray!6}{well-designed}\\

bien s'adapter & to fit\\

\cellcolor{gray!6}{blanchir} & \cellcolor{gray!6}{to bleach}\\

brillant & glossy\\

\cellcolor{gray!6}{carton} & \cellcolor{gray!6}{cardboard (U)}\\

choisir & to choose\\

\cellcolor{gray!6}{commander qqch} & \cellcolor{gray!6}{to order stg}\\

d'apparence professionnelle & professional-looking\\

\cellcolor{gray!6}{de la cire} & \cellcolor{gray!6}{wax (U)}\\

de la colle & glue (U)\\

\cellcolor{gray!6}{dorure à chaud} & \cellcolor{gray!6}{hot foil stamping}\\

dos carré-collé & perfect-binding\\

\cellcolor{gray!6}{décrire} & \cellcolor{gray!6}{to describe}\\

délavé & washed out\\

\cellcolor{gray!6}{dépenser (ou passer du temps)} & \cellcolor{gray!6}{to spend}\\

empiler & to stack\\

\cellcolor{gray!6}{encre} & \cellcolor{gray!6}{ink}\\

enfance & childhood\\

\cellcolor{gray!6}{enlever} & \cellcolor{gray!6}{to remove}\\

essayer de faire qqch & to try to do stg\\

\cellcolor{gray!6}{expédier qqch} & \cellcolor{gray!6}{to ship out stg}\\

fabriquer qqch & to manufacture stg\\

\cellcolor{gray!6}{faire du télétravail} & \cellcolor{gray!6}{to work from home}\\

glaçage & glazing\\

\cellcolor{gray!6}{hauteur} & \cellcolor{gray!6}{height}\\

l'amidon & starch (U)\\

\cellcolor{gray!6}{la couverture} & \cellcolor{gray!6}{the cover}\\

la graisse (typographie) & the weight\\

\cellcolor{gray!6}{la rogne} & \cellcolor{gray!6}{cut-offs}\\

largeur & width\\

\cellcolor{gray!6}{le dos d'un livre} & \cellcolor{gray!6}{the spine}\\

longueur & length\\

\cellcolor{gray!6}{lycée} & \cellcolor{gray!6}{high-school}\\

mat & matte\\

\cellcolor{gray!6}{mélanger (b...)} & \cellcolor{gray!6}{to blend}\\

mélanger (m...) & to mix\\

\cellcolor{gray!6}{obsolète} & \cellcolor{gray!6}{outdated}\\

ondulé (carton) & corrugated\\

\cellcolor{gray!6}{ondulé} & \cellcolor{gray!6}{wavy}\\

paresseux & lazy\\

\cellcolor{gray!6}{pelliculage} & \cellcolor{gray!6}{lamination}\\

permettre à qn de faire qqch & to enable sb to do stg\\

\cellcolor{gray!6}{permettre à quelqu'un de faire qqch} & \cellcolor{gray!6}{to allow sb to do stg}\\

piqûre à cheval & saddle-stitching\\

\cellcolor{gray!6}{profondeur} & \cellcolor{gray!6}{depth}\\

précédent & previous\\

\cellcolor{gray!6}{relier un livre} & \cellcolor{gray!6}{to bind a book}\\

reliure spirales & coil binding\\

\cellcolor{gray!6}{remarquer} & \cellcolor{gray!6}{to notice}\\

rugueux & rough\\

\cellcolor{gray!6}{réparer} & \cellcolor{gray!6}{to fix}\\

résumer & to summarize\\

\cellcolor{gray!6}{s'intéresser à qqch} & \cellcolor{gray!6}{to be interested in stg}\\

se débarrasser de qqch & to get rid of stg\\

\cellcolor{gray!6}{suivant} & \cellcolor{gray!6}{next}\\

sécher & to dry\\

\cellcolor{gray!6}{tard/en retard} & \cellcolor{gray!6}{late}\\

traiter de & to deal with (dealt x 2)\\

\cellcolor{gray!6}{traiter une commande} & \cellcolor{gray!6}{to run an order}\\

travailleur & hard-working\\

\cellcolor{gray!6}{télécharger} & \cellcolor{gray!6}{to download}\\

téléverser & to upload\\

\cellcolor{gray!6}{un appareil électronique} & \cellcolor{gray!6}{a device}\\

un autocollant & a sticker\\

\cellcolor{gray!6}{un blanchet} & \cellcolor{gray!6}{un blanket}\\

un bobine & a reel\\

\cellcolor{gray!6}{un bâtonnet (yeux)} & \cellcolor{gray!6}{a rod}\\

un calage & a make-ready\\

\cellcolor{gray!6}{un client} & \cellcolor{gray!6}{a customer}\\

un devis & a quote\\

\cellcolor{gray!6}{un dépliant} & \cellcolor{gray!6}{a folded leaflet}\\

un entrepôt & a warehouse\\

\cellcolor{gray!6}{un logiciel} & \cellcolor{gray!6}{a software}\\

un manchon & a shrinkable sleeve\\

\cellcolor{gray!6}{un massicot} & \cellcolor{gray!6}{a trimmer}\\

un métier, commerce & a trade\\

\cellcolor{gray!6}{un niveau} & \cellcolor{gray!6}{a level}\\

un nuancier & a fan deck\\

\cellcolor{gray!6}{un plateau} & \cellcolor{gray!6}{a tray}\\

un pli roulé (UK) & a roll fold\\

\cellcolor{gray!6}{un pli roulé (US)} & \cellcolor{gray!6}{a tri/letter fold}\\

un pli-fenêtre & a gate-fold\\

\cellcolor{gray!6}{un pli} & \cellcolor{gray!6}{a fold}\\

un rabat & a flap\\

\cellcolor{gray!6}{un revêtement} & \cellcolor{gray!6}{a coating}\\

un rouleau & a roller\\

\cellcolor{gray!6}{un stage} & \cellcolor{gray!6}{an internship}\\

un trait & a stroke\\

\cellcolor{gray!6}{un volume} & \cellcolor{gray!6}{a form}\\

un écran (d...) & a display\\

\cellcolor{gray!6}{un écran (m...)} & \cellcolor{gray!6}{a monitor}\\

un équilibre & a balance\\

\cellcolor{gray!6}{une caractéristique, une spécification} & \cellcolor{gray!6}{a feature}\\

une cellule & a cell\\

\cellcolor{gray!6}{une couche} & \cellcolor{gray!6}{a layer}\\

une encoche & a notch\\

\cellcolor{gray!6}{une entreprise} & \cellcolor{gray!6}{a company}\\

une fente & a slit\\

\cellcolor{gray!6}{une forme} & \cellcolor{gray!6}{a shape}\\

une lame & a blade\\

\cellcolor{gray!6}{une livraison} & \cellcolor{gray!6}{a delivery}\\

une machine empileuse & a stacker\\

\cellcolor{gray!6}{une nuance de couleur} & \cellcolor{gray!6}{a hue}\\

une page & a sheet\\

\cellcolor{gray!6}{une plaque} & \cellcolor{gray!6}{a plate}\\

une poignée & a handle\\

\cellcolor{gray!6}{une police (de caractères)} & \cellcolor{gray!6}{a font}\\

une rainure & a scoring line\\

\cellcolor{gray!6}{une récompense, un prix} & \cellcolor{gray!6}{a prize}\\

une usine & a factory\\

\cellcolor{gray!6}{une étape} & \cellcolor{gray!6}{a step}\\

vernis sélectif & spot varnish\\

\cellcolor{gray!6}{vif (couleur)} & \cellcolor{gray!6}{bright}\\

écorce & bark (U)\\

\cellcolor{gray!6}{épaisseur} & \cellcolor{gray!6}{thickness}\\

être bon dans qqch & to be good  at stg\\

\cellcolor{gray!6}{un propriétaire} & \cellcolor{gray!6}{an owner}\\

un réseau & a network\\

\cellcolor{gray!6}{difficile} & \cellcolor{gray!6}{tough}\\

une falaise & a cliff\\

\cellcolor{gray!6}{une vente} & \cellcolor{gray!6}{a sale}\\

acheter & to purchase\\

\cellcolor{gray!6}{améliorer, mettre en valeur} & \cellcolor{gray!6}{to enhance}\\

une légende (dessin) & a caption\\

\cellcolor{gray!6}{jetable} & \cellcolor{gray!6}{disposable}\\

une marque & a brand\\

\cellcolor{gray!6}{un outil} & \cellcolor{gray!6}{a tool}\\
\bottomrule
\end{longtable}

%\end{linenumbers}

\end{document}
